

\documentclass[11pt, a4paper, twoside, openright]{book}

\usepackage{a4wide}
\usepackage{amsmath,amssymb, amsthm} % AMS math
\usepackage{color}
\usepackage{enumerate}
\usepackage{fancyhdr}
\usepackage{fancyvrb}
\usepackage{floatflt}
\usepackage{multirow}


\usepackage{listings}
\usepackage{minitoc}
\usepackage{nextpage}
\usepackage{parskip}
\usepackage{setspace} 
\usepackage{xspace}

\usepackage{float}
\usepackage{subfigure}
\usepackage{subfloat}

\usepackage{makeidx}

\pdfcompresslevel=9

\setcounter{tocdepth}{1}
\setcounter{minitocdepth}{1}
\setcounter{secnumdepth}{3}


\pagestyle{fancyplain}
\renewcommand{\chaptermark}[1]%
        {\markboth{\chaptername\ \thechapter\ \ #1}{}}
\renewcommand{\sectionmark}[1]%
        {\markright{\thesection\ \ #1}}
\fancyhead[LE,RO]{\fancyplain{}{\thepage}}
\fancyhead[RE]{\fancyplain{}{\it\leftmark}}
\fancyhead[LO]{\fancyplain{}{\it\rightmark}}
\fancyfoot[C]{\fancyplain{\thepage}{}}                                         

\setlength{\headheight}{15pt} 

\newcommand{\thesischapter}[1]{\chapter{#1}\minitoc}

\usepackage{stmaryrd, hyphenat}
\usepackage{graphicx}
\usepackage{wrapfig}
\usepackage{verbatim} 
\usepackage{mathtools}
\usepackage{xspace}
\usepackage{bussproofs}
\usepackage{multirow}


\newcommand{\mybar}{\mbox{\Large \textbar}}
\newcommand{\mypause}{\pause}

% \newenvironment{proof}[1][Proof]{\begin{trivlist}
% \item[\hskip \labelsep {\bfseries #1}]}{\end{trivlist}}


\newcommand{\SCADE}{{SCADE\ }}
\newcommand{\scade}{{\small \sc \textit{SCADE}}}

% \newtheorem{theorem}{Theorem}[section]

\usepackage{tikz}
%%\usetikzlibrary{arrows,backgrounds,snakes,shapes}




\newcommand{\rungStart}{
  %% Draw a -
  -- ++(2,0)
}

\newcommand{\contact}[1]{
  %% Draw a -
  -- ++(1,0)
  %% Draw ]
  +(-0.1, 0.5) -- +(0, 0.5) -- +(0, -0.5)  -- +(-0.1, -0.5)
  %% Move across
  ++(0.4, 0)
  %% Draw text above
  +(0,0.8) node{#1}
  %% Move across
  ++(0.4, 0)
  %% Draw [
  +(0.1, 0.5) -- +(0, 0.5) -- +(0, -0.5)  -- +(0.1, -0.5)
  %% Reset to current point
  ++(0, 0) --
  %% Draw a -
  ++(1,0) 
}



\newcommand{\closedContact}[1]{
  %% Draw a -
  -- ++(1,0)
  %% Draw ]
  +(-0.1, 0.5) -- +(0, 0.5) -- +(0, -0.5)  -- +(-0.1, -0.5)
  %% Move across
  ++(0.4, 0)
  %% Draw text above
  +(0,0.8) node{#1}
  %% Draw close i.e. /
  +(-0.2, -0.5) -- +(0.2, 0.5)
  %% Reset to current point (i.e. middle)
  ++(0, 0)
  %% Move across
  ++(0.4, 0)
  %% Draw [
  +(0.1, 0.5) -- +(0, 0.5) -- +(0, -0.5)  -- +(0.1, -0.5)
  %% Reset to current point
  ++(0, 0) --
  %% Draw a -
  ++(1,0) 
}

\newcommand{\coil}[1]{
   %% Draw a -
  -- ++(1,0)
  %% Move to end of -
  +(0.5,-0.5)
  %% Draw arc (
  arc (270:90:0.5cm)
  %% Move across
  ++(0.3, 0)
  %% Draw text above
  +(0,0.3) node{#1}
  %% Reset
  +(0.3,-1)
  %% Draw Arc )
  arc (-90:90:0.5cm)
  %% Move to center of )
  ++(0.5,-0.5)
  %% Finish Path
  ;
}



\newcommand{\nodeLink}[1]{
  %% Draw a -
  -- ++(0.5,0) node[inner sep=0pt, minimum size=0pt] (#1){};
}

\newcommand{\nodeLinkInline}[1]{
  %% Draw a -
  -- ++(0.5,0) node[inner sep=0pt, minimum size=0pt] (#1){} -- ++(0.5,0) 
}


\newcommand\boxAngle{150}


\newcommand{\mybox}[5]{ 
  %% #1 = Width, #2 = Height, #3 = Depth, #4 = x, #5 = y
  %% Front face
  \draw  (#4,#5) rectangle +(#1,#2);
  %% Draw side
  \draw [fill=gray!25] (#4,#5) -- ++(\boxAngle:#3) -- ++(0,#2) -- ++(\boxAngle:-#3);
  %% Draw top
  \draw [fill=gray!50] (#4,#5) ++(0,#2) -- ++(\boxAngle:#3) -- ++(#1,0) -- ++(\boxAngle:-#3);
}

\tikzstyle{myArrow}=[draw,->, line width=1.5pt]

\tikzstyle{box}=[fill=blue!20,draw=blue!80,rounded corners,thick,text width=1.9cm,text centered,font=\tiny]
\tikzstyle{box3}=[draw,thick,text width=1.9cm,text centered,font=\tiny]
\tikzstyle{oval}=[box,fill=red!20,draw=red!80]
\tikzstyle{rdm}=[box,fill=green!20,draw=green!80]

\tikzstyle{to}=[->,thick,shorten >=1pt,shorten <=1pt]
\tikzstyle{from}=[to,<-]
\tikzstyle{train}=[line width=3pt,color=blue,>=fast cap,<-,densely dashed]

\usepgflibrary{arrows}
\usetikzlibrary{through} 

\def\kw#1{{\color{red}#1}}
\def\dull#1{{\color{black!75}#1}}


\newtheorem{mydef}{Definition}
\newtheorem{mytheorem}{Theorem}
\newtheorem{proposition}{Proposition}
\newtheorem{myremark}{Remark}
\newtheorem{proofbegin}{Proof}
\newtheorem{mylemma}{Lemma}
\newtheorem{myproof}{Proof}
\newtheorem{myexample}{Example}






\begin{document}




${ \ } $ \vspace{4cm}
%
%% Titlepage 1 
%

\begin{center}
{\huge \bf Application of Verified SAT algorithms to the Verification of Train Control Systems}
\end{center}

\begin{center}
{\large \bf Andrew Lawrence}
\end{center}

\begin{center}
June, 2014
\end{center}

\vspace{2.5cm}



\vfill

\begin{center}
A thesis submitted to Swansea University\\
in candidature for the degree of Doctor of Philsophy
\end{center}

\begin{center}
%%%includegraphics[scale=0.4]{logo} The logo file is currently missing
\end{center}

\begin{center}
Department of Computer Science\\
Swansea University
\end{center}

\thispagestyle{empty}

\newpage
\thispagestyle{empty}

\mbox{}

\clearpage



\thispagestyle{empty}

\section*{Declaration}

This work has not previously been accepted in substance for any degree
and is not being currently submitted for any degree.

\vspace{0.5cm}
\begin{tabular}{l}
June , 2014\\
\\
Signed:
\end{tabular}

\section*{Statement 1}
This thesis is being submitted in partial fulfilment of the
requirements for the degree of a MRes in Logic and Computation.

\vspace{0.5cm}
\begin{tabular}{l}
June , 2014\\
\\
Signed:
\end{tabular}

\section*{Statement 2}

This thesis is the result of my own independent
work/investigation, except where otherwise stated. Other sources are
specifically acknowledged by clear cross referencing to author, work,
and pages using the bibliography/references. I understand that failure
to do this amounts to plagiarism and will be considered grounds for
failure of this dissertation and the degree examination as a whole.

\vspace{0.5cm}
\begin{tabular}{l}
June, 2014\\
\\
Signed:
\end{tabular}

\section*{Statement 3}

I hereby give consent for my thesis to be available for
photocopying and for inter-library loan, and for the title and summary
to be made available to outside organisations.

\vspace{0.5cm}
\begin{tabular}{l}
February 18, 2011\\
\\
Signed:
\end{tabular}

\clearpage
\thispagestyle{empty}
\mbox{}

\newpage

\tableofcontents


\clearpage


 \thesischapter{Introduction}{Introduction}

This thesis is concerned with the application of formal methods within the Railway Domain and to the tools used when applying formal methods themselves. Firstly we present a new approach to develop verified SAT solving algorithms, which have been applied to the verification of a real world train control system: the solid state interlocking. Secondly we present an approach to formalise and model the European Rail Traffic Management System (ERTMS), in such a way as to capture the performance of the system and also apply a model checker to verify the system's safety.

\section{Motivation}

A major problem facing those who design and develop computer systems today is  that of designing and implementing such systems correctly. The answer to this problem is more elusive than it may seem initially due to the underlying complexity of modern systems. The process of checking that a system meets its specification is called \emph{verification}. In industry the most widely used means to verify both hardware and software systems is testing. This checks that for a given input the output of the system is correct with regards to the specification. The biggest downfall of testing is that it is typically used in a \emph{non-exhaustive} manner, due to the fact that most modern systems have a large number of inputs which makes testing each possible combination is not feasible. This problem is further compounded when several systems are operating in parallel and communicating with each other.

One of the most famous and costly errors to occur was that of the Ariane 5 flight 501\cite{GL97}. A 64-bit floating point number representing horizontal velocity was converted into an 16-bit signed integer resulting in an overflow and a hardware exception being raised. The control system interpreted the hardware exception as position and velocity data which lead it to fly on a self destructive trajectory. Another example of a flawed computer system is "Chip and PIN" smart cards secured by the EMV protocol \cite{JM10}. It is supposed to be the case that only someone with the 4 digit personal identification number (PIN) can make payments with the cards, however it is possible for a fraudster to perform a man-in-the-middle attack and to trick the terminal into believing the PIN was correctly entered when no PIN was entered at all. With over 730 million smart cards in circulation this posses a significant threat the to security of many peoples finances. Both of these problems were avoidable. In the case of the Ariane rocket all that would have been required was an extra line of code or two to prevent the variable over-flowing. Many other variables in the code had been protected against over-flow in this way; it was simply over looked in the case of this particular variable.


Formal methods are a group of approaches that capture certain aspects of the development process with a logical and mathematical basis. This thesis is mostly concerned with \emph{formal verification}, which uses logic and formal reasoning to prove that a program meets its specification, and \emph{formal specification}, which attempts to capture the behaviour of the program in a logic or language with a logical underpinning. The advantage of using formal verification in combination with formal specification is that one can prove that a system satisfies its specification for all inputs in an \emph{exhaustive} manner. These formal methods however do not make testing redundant, for example, e.g. would you rather fly a plane that has been formally verified and never tested or a plane that has been heavily tested but not verified? Another problem is that manual formal verification requires a large amount of technical ability to perform and in addition is time consuming. There are a number of automatic verification techniques, however these are typically not complete in the sense that they are not guaranteed to produce an output for every possible input. One area that automatic techniques do perform well in is verifying large but logically un-complex systems, which do occur in industry. It is also beneficial if these systems are part of a product line. If that is the case then a customised verification approach may be developed that exploits similarities between individual implementations of the system.

One such automatic verification technique is \emph{model checking} which attempts to exhaustively search the state space of the system and check that a safety property holds in each individual state. This can be done by formulating the model checking problem as a boolean satisfaction problem and then applying a SAT solver to do the search. These SAT solvers have the advantage that they are highly optimised and are typically faster at solving any boolean satisfaction problem, without any problem specific customisation, than any proprietary software. However, low level and complex nature of many SAT solver optimisations makes it harder to reason about the solver's correctness. This causes a problem when such solvers are used in a safety critical environment where their results must be trustworthy. Besides the correctness also
totality (or universality) of SAT solvers is an issue. For example, in the 2012 SAT competition (www.smtcomp.org) many systems were not total in the sense that they returned
"Unknown" for certain inputs signifying that they could not deal with the given problem.  There are also numerous specialised model checking algorithms, each of which, is geared towards solving a specific type of problem and comes with their own unique optimisations.


Another approach to formal verification is that of interactive theorem proving \cite{HG09,SSD06} where a tool aids with the human driven construction of a formal proof of correctness for a system. These tools have the advantage that the machine can check the human produced proof and they typically include a number of built in tactics that aid in the construction of a proof. Some interactive theorem provers realise the Curry-Howard correspondence \cite{HC34, HC58, WH80} which states that constructive proofs performed using the natural deduction calculus can be viewed as programs in the lambda calculus.  The process of exploiting this correspondence to produce a program from a proof is called program extraction. This technique allows for the development of programs that are correct by construction as the program comes with a proof that certifies its correctness. The Minlog theorem prover comes with a number specialisations for the purpose of extracting programs from formal proofs.
\begin{comment}
\textbf{Note: Add something about program extraction here.}
Alternatively to automatic techniques for verification there are also interactive theorem provers which employ man-machine collaboration in order to prove properties over a system.
Program extraction is another verification technique which allows the production of correct by construction computer programs. It is based around the Curry-Howard correspondence \cite{} which states that constructive proofs performed using the natural deduction calculus can be viewed as programs in the lambda calculus. 
\end{comment}


The Railway Verification Group at the Computer Science Department of Swansea University has been researching the development of formal methods for use in real world situations in particular for the development of train control systems. The majority of this work has been in the form of an industrial collaboration with our partner Siemens Rail Automation (UK). Siemens are continuously modernising and improving their development processes and as a part of this are looking to use formal methods for the development of their systems. The Westrace interlocking is one such system that is being considered for treatment by formal methods. These interlockings have the benefits of being large but relatively un-complex and developed has to comply to the same range of safety requirements.  Another product currently under development by Siemens Rail Automation is the European Rail Traffic Management System (ERTMS). This system is developed to a European  standard with each country or individual company producing their own implementation. Since the specification is loose and allows for a wide range for implementations this makes modelling of the system desirable to understand possible behaviours and to ensure safe integration with existing infrastructure. Furthermore, it is an open field  when it comes to substantiating the claim that the ERTMS approach offers a higher performance of the railway compared to the traditional control systems.




\section{Results of the Thesis}
 

The first result of this thesis (Chapter 4/5) is the use of program extraction to obtain a verified DPLL SAT solving algorithm that can solve industrial sized problems in the form of verifying solid state interlockings following the approach set out in \cite{AL14a}. We have named the resulting SAT solver X-SAT and show that its performance is comparable with other verified SAT algorithms. This case study opens up a new area of application for program extraction to obtain verified SAT and resolution algorithms and has been published in \cite{AL12,AL14b}. This is one of the largest case studies so far in the area of program extraction and demonstrates that it can be used for practical purposes.  One of the biggest arguments against the use of program extraction is that one loses control over the efficiency of the program. We have demonstrated that it is possible to carry out efficiency considerations at the proof level and therefore obtain a more efficient extracted program. We demonstrate that it is possible to modify the SAT proof system and extract a more advanced SAT algorithm that implements an optimisation called clause learning. We also demonstrate that it is possible to extract a resolution algorithm based on our original DPLL SAT solver.


The second result of this thesis is the modelling of the European Rail Traffic Management System and verification of its safety and performance using Real Time Maude. First, we formalised the ERTMS system, controlling an example railway in the form of a pentagon, as several hybrid automata in order to capture its behaviour.  This pentagon example along with a small junction were then modelled as Real Time Maude specifications. In our model, speed, acceleration, braking behaviour and track length are integral parts, as even safety cannot be decided without these properties. The inclusion of these properties in the model enables the performance of this system to be analysed by executing the Real Time Maude specifications. This is also the first formalisation of ERTMS to capture the interlocking, radio block processor, train and the communications between them. The interface between the interlocking and the radio block processor, in particular, has yet to be examined in detail even though it plays a crucial role in the safety of the overall system.

\section{Publications}
The follow is a list of publications that I have produced during my studies

\textbf{Verification of Railway Interlockings in SCADE (extended abstract with Seisenberger), AVOCS 2010 \cite{AL10}} 

\textbf{Verification of Solid State Interlocking Programs (with Chadwick, James, Kanso, Moller, Roggenbach and Seisenberger), FM-Rail-BOK, \cite{AL14a}}

\textbf{Extracting a DPLL Algorithm (with Berger and Seisenberger), MFPS 2012, \cite{AL12}}

\textbf{Extracting Verified Decision Procedures: DPLL and Resolution (with Berger, Nordval Forsberg and Seisenberger), LMCS accepted for publication \cite{AL14b}}

\textbf{Safety and Performance of the European Rail Traffic Management System: A Modelling and Verification Exercise in Real Time Maude (with Berger, Roggenbach and Seisenberger), WADT 2014 \cite{AL14c}}


\section{Thesis Outline}

%%\textbf{Chapter 1}: This introduction. 
%%\medskip \\
\textbf{Chapter 2}: Provides an overview of the formal foundations of the Minlog interactive theorem prover and program extraction. This is background information.
\medskip \\
\textbf{Chapter 3}: Provides an overview of SAT solving, including the DPLL algorithm and  proof system, clause learning and resolution. This is background information.
\medskip \\
\textbf{Chapter 4}: Describes the proof of soundness and completeness for the DPLL and resolution proof systems. The completeness proofs are new.
\medskip  \\
\textbf{Chapter 5}: Describes the formalisation of the DPLL and Resolution proof systems along with their soundness and completeness in the Minlog System. The proof of completeness for the DPLL proof system forms the basis of our extracted SAT solver X-SAT. The final part of this section describes the behaviour of the extracted program X-SAT which is compared with another verified solver. The formalisation and extraction presented in this section are new results.
\medskip \\ 
\textbf{Chapter 6}: Describes an approach to extract a prototype conflict driven clause learning SAT algorithm by modifying the DPLL proof system. This is a new formalisation.
\medskip \\
\textbf{Chapter 7}: Provides a general introduction to the traditional railway domain. This is background information.
\medskip \\
\textbf{Chapter 8}: Provides an overview of the model checking techniques used in this thesis for verification. This is background information. \medskip \\
\textbf{Chapter 9}: Provides a description of the European Rail Traffic Management System. This is background information.
\medskip \\
\textbf{Chapter 10}: Describes the formalisation of the European Rail Traffic Management System as 3 hybrid automata. This is a new formalisation.
\medskip \\
\textbf{Chapter 11}: Describes how the European Rail Traffic Management System can be modelled as a Real Time Maude specification  specification. This is a new modelling approach.
\medskip \\
\textbf{Chapter 12}: Validates the Real Time Maude specification of ERTMS using simulation via execution of the specification and describes the verification of the specification using the Maude LTL model checker. This is a new simulation and verification approach.


\medskip

\begin{comment}

\begin{itemize}

\item Formalize the DPLL proof system and a proof of its completeness.

\item Extract a standard DPLL SAT algorithm from the formalisation into a functional programming language and test its performance

\item Modify the formalisation and completeness proof  of the DPLL proof system so that it captures the behaviour conflict driven clause learning SAT algorithms.

\item Extract a clause learning SAT algorithm from the formalisation  of the modified DPLL proof system and show that clause learning increases the efficiency of the solver.

\item Apply verified SAT algorithms to the verification of solid state railway interlocking programs. 

\end{itemize}

Our second set of results is in regard to the formal specification and verification of the European Rail Traffic Management System (ERTMS) using Real Time Maude.

\begin{itemize}

\item Formalise ERTMS as a hybrid automata.

\item Formalise and Model ERTMS as a Real Time Maude specification.

\item Verify ERTMS using Real Time Maude's linear temporal logic model checker.

\end{itemize}


\end{comment}

\vspace{5cm}

\section{Hardware}
The following computer was used for all benchmarks and tests run as part of this thesis:
\begin{center}
\begin{tabular}{| l | c |} \hline
Operating system: & Ubuntu 14.04 LTS  \\ \hline
Processor: & Intel(R) Core(TM) i7 64 bit CPU with 8 cores at 2.93GHz  \\ \hline 
Memory: & 8 GBs 1333 MHz DDR2 RAM \\ \hline
Hard disk: &  750GB 7200 RPM  SATA\\  \hline 
\end{tabular}
\end{center}

\section{Implementations}
The Minlog implementations described in Chapters \ref{chapter:dpllminlog} and \ref{chapter:cdclproof} and Real Time Maude implementations described in \ref{chap:modelertms} and \ref{chap:verifyertms} are available at \url{http://cs.swan.ac.uk/~csal/phd/index.html}.

\section{Acknowledgements}
There are a great many who have supported me or contributed in some way throughout the writing of this thesis. I would first and foremost like to thank my supervisor Monika Seisenberger for her eternal patience and guidance. She introduced me to both of the tools used in this thesis.


Siemens Rail Automation/ Invensys Rail deserve many thanks for their support and valuable technical input throughout the whole of my research career.
 

My mother Sharon Lawrence provided me with many cooked meals and washed my clothes during my stay at her house; this enabled me to spend even more of my time on the writing of this thesis.



Fredrik Nordsval Forsberg has made several contributions to this thesis. Firstly, he wrote the automatic translation which enabled Minlog terms to be translated into Haskell.  I would have been unable to perform such an in-depth analysis of the extracted program without it. He was also a co-author and worked 

Ulrich Berger has provide his expert guidence numerous times both as an author

Markus Roggenbach introduced me to Hybrid Automata and provided valuable knowledge on the Railway Domain.

Alison Jones for sharing her in depth understanding of the English language and for proof reading my thesis.






My lovely girl friend Cerri has supported with m




\thesischapter{Background: Traditional Railway Control Systems:}
The birth of the railways occurred in the 1800s and from that point until the present day they have experienced continual growth, change and development. In the beginning the burden of safety was placed solely on human shoulders and has since been placed on mechanical systems before finally being transferred to electronics systems that are in place guaranteeing safety today. In the following we shall present a brief history of the railway followed by information on our industrial partner Invensys Rail.  Following this general introduction we shall look into specific equipment found on the modern railway and other information needed to understand the domain. Most importantly we shall describe the Westrace interlocking, a modern system charge with guaranteeing safety on the railway, which is produced by Invensys. In the final part of this chapter we shall look at some previous work in this field.


\begin{comment}
From their birth in the 1800s to the present day, the railway and its control
systems have seen many advances. Its control and safety has gone from being a
completely manual human based system, to a mechanical system and finally to the electronic
system we see today. We will now look at a brief history of the railway
followed by information on our industrial partner Invensys Rail. We then look
more closely at modern railways and the equipment which constitutes
them. We also study
Westrace interlocking which is produced by Invensys and the
ladder logic programs which run on it. Finally, we look at some previous work
in this field.
\end{comment}

\section{ A History of Railway Signalling and Control Systems}
Initially in the early days of the railway there was not fixed signals as we see today. Instead Policemen stationed at junctions and railway stations, were charged with signalling train drivers using a system of flags or oil lamps and changing the points of the railway. A major problem of the time, without telecommunications, was that there was no way of locating a train once it went out of sight. This meant that in practice the only safety precaution that could be taken was to delay the departure of trains using an egg timer which would hopefully space out their positions along the track. The level of safety resulting from this precaution was acceptable as train speeds were not relatively high at the time.


\begin{comment}
Prior to the days of fixed signals, Policemen would be stationed at
junctions and railway stations. They changed points manually and gave instructions to train drivers
by using a system of either flags or oil lamps depending on the
visibility. Since this was before the time of telecommunications and
electricity there was no way of telling where a train was once it left a
station and went out of sight. The only safety precaution that could be taken
was to use an egg timer to delay the departure of the next train in order to
give the previous train time to progress along the track. Train speeds were
not very high during this period so this was an acceptable way of ensuring safety.
\end{comment}


The most important type of signal found in the modern railway is the \textbf{fixed signal} which are static, positioned by the side of the track and visually present information to the train driver. The first fixed signals were wooden boards shaped in such a way to provide specific information attached to poles which rotate around a vertical axis. Typically these boards would form a signal instructing the train driver to stop, however if the board was rotated side-on to the driver then it becomes inactive and the driver would proceed if it is safe to do so.

\begin{comment}
Modern railway signally makes use of \textbf{fixed signals}. These are permanently positioned
by the side of the track and provide some visual information to the train driver.
The original fixed signal consisted of a shaped wooden board that could be
rotated on pole round a vertical axis. If the board was visible to the driver
then he would have to stop the train. On the other hand if the driver couldn't
see the board because it was side-on to him then he would be able to
proceed.
\end{comment}

The next major enhancement of these signals came in the form of the \textbf{Semaphore} fixed signal.  Instead of having 2 positions (visible/ not visible), the boards of a semaphore signal could be moved into one of several predetermined positions. The most common of these signals had 3 visible positions or aspects in which they could be placed which would rely the following information to an approaching driver: proceed if its safe, proceed with caution if its safe and stop.

\begin{comment}
One of the major developments in railway signalling was the introduction of
the \textbf{Semaphore} fixed signal. These consisted of a board that could be
moved into several preset positions. Typically these would have  3 different visible
``aspects'' which they could be set to: One aspect to indicate the driver can
proceed, another that indicates the driver can proceed with caution and
finally an aspect which indicates that the driver should stop. 
\end{comment}

The introduction of the semaphore signal also foreshadowed another major change in the system for controlling the signals behind the scenes.  A new band of professional \textbf{Signallers} were employed to specifically manage the railways, replacing the Policemen that proceeded them in the progress. The efficiency of signal control was also improved by a system of wires, pulleys and levers which enabled multiple signals and points to be controlled from a central position. This central position was typically enclosed in a box giving rise to the name \textbf{signal box} which were typically manned by one or more signallers. The controlling signals and points from the centrality of the signal box also enabled for more safety mechanisms to be installed. The most prominent of these the \textbf{interlocking} shall be the subject of interest to us in later chapters. The interlocking physically prevented the control system of a railway from unsafe state by physically locking levers.


\begin{comment}
Around about the same time as the introduction of the semaphore signal, the system
for controlling the signals went under drastic change. The Policemen were
replaced with professional \textbf{Signallers} whose job was specifically to
manage the railways. A system of pulleys, wires and levers was also devised
to allow multiple signals and points to be controlled from a central
position. This central position became known as a \textbf{signal box} and was
manned by one or more signallers. This centralisation allowed for further
safety mechanisms to be installed. One in particular, namely the \textbf{interlocking},
is of interest to us. The interlocking physically locked levers if they were
unsafe to move.
\end{comment}
The advent of electricity brough about a further advance in railway technology, allowing for the development of the \textbf{track circuit}. When a train was on a segment of track it completed the circuit between the two rails and light up an indicator in the signal box. As more track circuits were deployed it became possible to control certain signals without human intervention. These \textbf{automatic signals} were completely operated by track circuits and prevented trains from entering a segment of track behind a signal already containing a train. Electric motors enabled the construction of electric point machines enabling the signaller to electronically control a point at a great distance with little physical effort. The area control by each signal box was greatly increased along with the number of signals controlable by one signaller. Electromechanical \textbf{relays} were introduced as a replacement for the soley mechanical relays reducing the size of each signal box and its footprint on the landscape.
\begin{comment}
The next leap in railway technology came from the invention of the electronic
\textbf{track circuit}. These would activate an indicator in the signal box if a
segment of track was occupied by a train. As more and more track circuits
became installed it was no longer necessary to have human intervention to
control certain signals. \textbf{Automatic signals} were introduced which
operated completely by track circuits without any intervention from human
signallers. Around this time \textbf{electric point machines} were introduced
removing a large amount of physical work performed by signallers allowing for
a greater area of control for each signaller.  Around this time
electromechanical \textbf{relays} began to replace purely mechanical relays
reducing the amount of space needed for a signal box.
\end{comment}

Another instance of an electrical system replacing a mechanical one occured in the 1920s. New \textbf{colour light signals} replaced  the mechanical semaphore signals and had the advantage of being brighter than the oil lamps of that signal.


\begin{comment}
In the 1920s \textbf{colour light signals} replaced mechanical semaphore signals these where much brighter than the oil
lamps fitted to semaphores and greatly increased the safety of night time
train travel. In the 1930s the mechanical levers were replaced with an electronic \textbf{control panel}
containing switches and buttons. This allowed for the introduction of
\textbf{route setting} where with the press of a button configurations of signals and points would be
associated with a particular route could become activated. Prior to this time
many levers would have had to have been pulled to set many different pieces of equipment.
During the 1980s the most important advance from our point of view took
place. The advent of electronic microprocessors enabled the replacement of the
relay and mechanical interlockings with an electronic \textbf{solid state interlocking}
system (SSI) \cite{AC08}. The main focus of this project will be to investigate the safety
of such solid state interlockings.
\end{comment}
\section{Invensys Rail}

Invensys Rail \cite{Inven} and its previous incarnation Westinghouse Rail Systems Ltd have been
involved for over 140 years in producing equipment to increase safety in the
railway industry. Originally they produced air brakes for trains, these had a
failsafe state such that if the power was cut the brakes would automatically stop the train.
Later on in the company's development they provided support to British Rail
when the first solid state digital railway interlocking was installed in
Leamington Spa. Today they supply railway control equipment to companies based
around the globe, including companies based in Australia, Hong Kong,
Germany, Spain and the UK. This project is mainly concerned with one of the
solid state railway interlockings Invensys produces called the Westrace. The
Westrace railway interlocking continuously runs a ladder logic program which
prevents the railway control systems from entering a dangerous state. Ladder
logic will be explained in a later chapter.
David Kerr and Tony Rowbotham  produced a book that explains the
terminology and methodology used in the railway industry and by
Invensys (See \cite{KR01}).  

\section{An Overview of the Railway Domain} 
In this section we present the features of the railway domain that are in
the scope of this thesis. We hope to provide the reader with the background
information and terminology necessary to understand the parts of this thesis.


\subsection{The Railway Topology: Track and Points}
We will now present an overview of the physical railway from a topological
point of view. To do this we will present an example of a small track plan
of a junction.  If the reader is interested in learning more about the
topology of the railway a more detailed description can be found in \cite{KR01} 


\begin{figure}

\begin{center}

\begin{tikzpicture}

\node at (-4 , 1.3)  {$A$};

\node at (-0.5 , 2) {$B$};

\node at (2.5 , -1) {$C$};

\node at (2.5 , 1.3) {$D$};


\draw[<->] (-1.5, 1.3) to node [above, sloped] {Normal} (0.5 , 1.3) ;

\draw[<->] (-1.5 , 0.3) to  node [below, sloped] {Reverse} (0.5, -0.5);


\draw (-5, 1) -- (3.4,1);




\draw [dashed] (-3 , 1.1)  -- (-3, 0.4);

\draw [dashed] (1.5 , 1.1) -- (1.5 , 0.4);

\draw [dashed] (2.2 , 0.075) -- (1.8, -0.675) ;

\draw (-2, 0.975) -- (-1, 0.975);

\draw (-1, 0.975) .. controls (0, 0.95) and (0.2 , 0.75)  .. (1, 0.4);

\draw (1, 0.4) -- (3.4, -0.65);



\draw (-5, 0.5) -- (3.4, 0.5);

\draw( -2, 0.475) -- (-1, 0.475);

\draw (-1, 0.475) .. controls (0, 0.45) and (0.2 , 0.25)  .. (1, -0.1);

\draw (1, -0.1) -- (3.4, -1.15);





\end{tikzpicture}

\end{center}
\caption{A Typical Junction}

\label{fig:track}

\end{figure}


\subsubsection{Track Segments}
A section of track is typically broken down into track segments each
containing one or more track circuits to detect the presence of a
train. Typically track segments become larger on long straight stretches of
track without any interesting topological features such as junctions or
stations. Likewise track segments become smaller around junctions and stations
where control over train movement is of greater importance. 

\subsubsection{Points}
A point is a physical piece of equipment that is used to 
form a junction. Due to the nature of the rails and trains it is not possible to physically
to just join two segments of track. Instead a point is needed to act as
physical switch controlling the flow of trains through a junction. A point
has two positions which are referred to as \textbf{normal} and
\textbf{reverse}. This presents a safety hazard, for example see figure 2.1, if a train enters
the junction $b$ from $c$ when the junction is locked in the
position for normal then the train will be derailed. 



\subsection{Railway Signalling}
Signals are the main means used to communicate information regarding the state
of the track ahead of the train. Typically they are placed either on the track
side or over hanging the railway. Visual indications known as aspects are used
to convey information to the driver. A signal will have many such aspects
which can be displayed, each with a particular meaning. The main type of signal
considered in this project is the coloured light signal. Typically these have
between one - four aspects each conveying a different indication about the state of
the track ahead. Below is a description of the
aspects used for a three light signal.  


\begin{figure}[h!]
\begin{description}

\item[Green] - If this aspect is displayed it indicates that track ahead is
  clear for a sufficient distance and the train driver
  can proceed at full speed to the next signal.

\item[Yellow] - This aspect indicates the track immediately ahead in between
  this signal and the next is clear however the driver should proceed with
  caution as a train could be in the track after that.

\item[Red] - This aspect indicates that the track ahead is not clear, the
  driver should stop and wait at this signal.

\end{description}
\label{fig:3lightsignal}


\end{figure}

The one aspect signal is typically a fixed red indicating that is not possible
to proceed down the track at this current point in time. The two - four aspect
signals are used on tracks with different speeds to convey different stopping
distances.  The two aspect signal for instance would be used on a low speed
track segment where stopping distances are relatively short. Whereas the four
aspect signal would be used on a high speed line where stopping distances are
long and the driver needs information for a greater length of track. These
signalling schemes are fixed in the UK however they are not fixed from country
to country. On the continent, for example, they may use different conventions,
colours and number of lights on each signal.




\subsection{The Westrace Interlocking}

The railway interlocking is a key component in ensuring the safety of the
railway. Its job is to apply a set of rules to the requests and commands it receives from the
control system and check whether or not the future state of the railway is safe.
If the control signals it receives do not violate the safety of the railway
then these signals are committed to the physical infrastructure. For example if
the human controller requests for a route to be set the interlocking will
process this request and ensure that it does not conflict with other routes
before allowing the command to be passed to the physical railway.

\begin{figure}[h!]

\begin{center}
\begin{tikzpicture}[node distance = 2cm]

\tikzstyle{box2}=[box3,text width= 5cm,font=\small]
\tikzstyle{arrow}=[->,very thick,shorten >=7pt,shorten <=7pt]


\node (A) [box2]                   {\textbf{Control System}\\
                                           };

\node (B) [box2, below of = A]              {\textbf{Railway Interlocking}\\
                                             };

\node (C) [box2, below of = B]              {\textbf{Physical Railway} \\
                                             };   


([xshift=-1cm]Sensor.south)
\draw [arrow] ([xshift=-1cm]A.south) -- node[sloped] {} ([xshift=-1cm]B.north);
\draw [arrow] ([xshift=1cm]B.north) -- node[sloped] {} ([xshift=1cm]A.south);
\draw [arrow] ([xshift=-1cm]B.south) -- node[sloped] {} ([xshift=-1cm]C.north);
\draw [arrow] ([xshift=1cm]C.north) -- node[sloped] {} ([xshift=1cm]B.south);

\end{tikzpicture}
\end{center}

\caption{The Location of the Railway Interlocking}
\label{fig:trackstate}
\end{figure}



The railway interlocking repeatedly executes a program or set of rules over some discrete time
interval. Each time it uses the set of
rules it contains to process a new set of inputs before committing them as
outputs. The Westrace interlocking used by Invensys Rail executes a so-called
ladder logic program to perform this process. 
The following are the three main stages of operation in the running of an
Westrace interlocking.

\begin{description}

\item[Reading of Inputs] - Read inputs from the control systems as well as the physical railway infrastructure.

\item[Internal Processing] - Execute the ladder logic program with the above inputs and calculate outputs.

\item[Committing of Outputs] - The outputs calculated in the previous cycle are then passed on to various places including the physical railway.

\end{description}




\section{Ladder Logic}

The Westrace Interlocking performs calculations by executing a ladder logic
program \cite{IEC03}. In the following we will look in more detail at these
ladder logic programs. The main concepts behind their construction and
behaviour will be presented. In later chapters we will provide a formal framework
for the verification of these programs.

The international standard for programmable logic controllers IEC
61131 \cite{IEC03} describes the graphical language ladder logic. It gets its
name from its graphical ``ladder'' like appearance which was chosen to suit
the control engineers responsible for their design. Each rung of the ladder is
used to compute an output variable from one or more input variables in the rung.
In the railway industry these input variables are referred to as contacts and
the output variables are referred to as coils. A description of the entities representing these variables
is as follows:


\smallskip

\begin{description}

\item[Coils]: These are used to represent values that are both stored for later use
  and output from the program. The value of a coil is calculated when a rung
  fires making use of the current set of inputs, the previous set of outputs
  and any outputs already computed for this cycle. The coil is always the
  right most entity of the rung and its value is computed by executing the
  rung from left to right.


\item[Open Contacts]: This entity represents the value of an un-negated variable


\item[Closed Contacts]: This entity represents the value of a negated variable.

\end{description}

\medskip

\begin{figure}[h!]
 \begin{center}
   \subfigure[A coil]{   \begin{tikzpicture}[scale=0.65, transform shape]
     \draw (0,-0) -- (0,-2);
     \draw (0,-1)  \rungStart \coil{C};

   \end{tikzpicture}
}
   \subfigure[An open contact]{   \begin{tikzpicture}[scale=0.65, transform shape]
     \draw (0,-0) -- (0,-2);
     \draw (0,-1)  \rungStart \contact{C};

   \end{tikzpicture}

}
    \subfigure[A closed contact]{   \begin{tikzpicture}[scale=0.65, transform shape]
      \draw (0,-0) -- (0,-2);
       \draw (0,-1)  \rungStart \closedContact{C};

    \end{tikzpicture}
  }
\end{center}
\label{fig:pelicanladder}
\caption{The Entities Used In Ladder Logic}

\end{figure}

\medskip

A Ladder logic rung is built using these entities and connections between
them. The shapes of the connections between the contacts determines how the
value of the coil is computed from them. Using propositional logic for comparison,  
a horizontal connection between two contacts represents logical conjunction
and a vertical connection between two contacts represents logical
disjunction see Figure \ref{fig:ladderconnectives}. 

\medskip

\begin{figure}[h!]
 \begin{center}
   \subfigure[$x \wedge y$ Conjunction]{   \begin{tikzpicture}[scale=0.65, transform shape]
     \draw (0,-0) -- (0,-2);
     \draw (0,-1)  \rungStart \contact{x} \contact{y} ;

   \end{tikzpicture}
}
   \subfigure[ $x \vee y$ Disjunction]{   \begin{tikzpicture}[scale=0.65, transform shape]
     \draw (0,-0) -- (0,-4);
     \draw (0,-1)  \rungStart \contact{x} \nodeLinkInline{A};
     \draw (0,-3)  \rungStart \contact{y} \nodeLink{B}

      \draw (A) -- (B);
   \end{tikzpicture}

}
\end{center}

\label{fig:ladderconnectives}
\caption{Logical Connectives In Ladder Logic}

\end{figure}

\medskip

In section 4 an approach is presented to capture the semantics of ladder logic
programs using propositional logic.


\section{Previous Work in this Field}
Previously James carried out work for Invensys, applying various SAT and model
checking techniques to verify the correctness of a simple pelican crossing and
two existing railway interlockings consisting of approximately 500 rungs (see
James \cite{PJames}).

James used  Kanso's work (See \cite{KKanso}), in particular his 
translation from ladder logic into propositional logic,
and applied several model checking techniques in order to try and
reduce the complexity of the problems. Both the work by Kanso and James was based on an early feasibility study by
Fokkink and Hollingshead \cite{WF98}. The relationship between a ladder logic program and
propositional logic was discussed in great detail. A method for formulating
such a ladder logic program as a formula in propositional logic was
presented. This laid the ground work for all successive projects involving
ladder logic. The possible application of program slicing was discussed and
this was later applied in the work by James \cite{PJames}.

Some of the techniques applied by James to the verification of ladder logic
are discussed below.

\begin{description}


\item[Bounded model checking:] This was the  main topic of the work of Phil James. It had the advantage that it produced counter
  example traces which are highly valuable to the engineers at Invensys. It allowed for the
  verification of 2000 iterations of the ladder logic programs provided
  without programming slicing and up to 20000 iterations of ladder logic
  programs with program slicing.

\item[Temporal Induction:] This is another technique used in the verification
  of the ladder logic programs, it succeeded whenever the inductive verification method
  Kanso applied also succeeded. It should however be stronger than inductive
  verification but no example was found to prove this. Temporal induction produced a counter
  example whenever the bounded  model checking produced a counter example. 

\item[Program Slicing:] This technique was combined with application of
  bounded model checking to reduce the state space requiring
  verification. This reduced the number of rungs in a ladder logic program
  by up to a factor of 10.

\end{description}







\bibliographystyle{alpha}
\bibliography{disbib}

\appendix


\thesischapter{Proof Systems}

\section{Gentzen's Sequent Calculus}
These rules and propositions were taken from St{\aa}lmarck \cite{GS00}.
\label{def:sequent}
\begin{mydef}[Sequent]
Every line in sequent calculus proof is called a sequent and takes the
following form. 

$$ A_1, \ldots , A_k \vdash B_1, \ldots ,B_l$$

In this text we are using the symbol $\vdash$ to represent the sequent arrow.
\footnote{The sequent arrow is sometimes represented using the symbol $\to$}
The following formula captures the meaning of the above sequent.

$$ \bigwedge_{i = 1}^{k} A_i \supset  \bigvee_{j-1}^l B_j$$

\end{mydef}
 
\medskip

Intuitively you can read a sequent as: If the conjunction of all of the $A_i$s
is true then one of the $B_j$s in the disjunction must be true.

\medskip

\begin{mydef}[Gentzen's Sequent Calculus PK] \hspace*{\fill} \\

Axiom
\bigskip
\begin{center}
$A \vdash A$
\end{center}

\bigskip

Structural Rules

\bigskip
\begin{center}


\AxiomC{$\Gamma \vdash \Delta $}
\LeftLabel{($Thinning$)}
\UnaryInfC{$\Gamma, \Theta \vdash \Delta, \Lambda $}
\DisplayProof \
\AxiomC{$\Gamma, A \vdash \Delta$}
\AxiomC{$\Gamma \vdash \Delta, A$}
\LeftLabel{($Cut$)}
\BinaryInfC{$\Gamma \vdash \Delta$}
\DisplayProof

\end{center}

\bigskip

Operational Rules

\bigskip
\begin{center}
\AxiomC{$\Gamma, A \vdash \Delta$}
\AxiomC{$\Gamma, B \vdash \Delta$}
\LeftLabel{($Or_L$)}
\BinaryInfC{$\Gamma, A \vee B \vdash \Delta$}
\DisplayProof \
\AxiomC{$\Gamma \vdash \Delta, A,B$}
\LeftLabel{($Or_R$)}
\UnaryInfC{$\Gamma \vdash \Delta , A \vee B$}
\DisplayProof

\end{center}

\smallskip

\begin{center}

\AxiomC{$\Gamma, A, B \vdash \Delta$}
\LeftLabel{($And_L$)}
\UnaryInfC{$\Gamma, A \wedge B \vdash \Delta$}
\DisplayProof \
\AxiomC{$\Gamma \vdash \Delta, A$}
\AxiomC{$\Gamma \vdash \Delta, B$}
\LeftLabel{($And_R$)}
\BinaryInfC{$\Gamma, \vdash  \Delta, A \wedge B$}
\DisplayProof

\end{center}

\smallskip

\begin{center}

\AxiomC{$\Gamma \vdash \Delta, A$}
\AxiomC{$\Gamma , B \vdash \Delta$}
\LeftLabel{($Imp_L$)}
\BinaryInfC{$\Gamma, A \to B \vdash \Delta$}
\DisplayProof \
\AxiomC{$\Gamma ,A \vdash \Delta , B$}
\LeftLabel{$(Imp_R)$}
\UnaryInfC{$\Gamma \vdash \Delta, A \to B$}
\DisplayProof


\end{center}

\smallskip

\begin{center} 

\AxiomC{$\Gamma \vdash \Delta , A$}
\LeftLabel{$(Neg_L)$}
\UnaryInfC{$\Gamma, \neg A \vdash \Delta$}
\DisplayProof \
\AxiomC{$\Gamma , A \vdash \Delta$}
\LeftLabel{$(Neg_R)$}
\UnaryInfC{$\Gamma \vdash \Delta, \neg A$}
\DisplayProof


\end{center}

\smallskip

\end{mydef}

\medskip

\begin{proposition}[Sub-formula Principle]
If a PK-proof P does not contain any application of the cut rule, then
All of the formulas occurring in  P must be a sub-formula of some
formula in the end sequent of P.
\end{proposition}

\medskip

Having the \textit{sub-formula principle} allowed St{\aa}lmarck to place bounds on the
size of proofs created by his algorithm.

\medskip

\begin{proposition}[Removing Thinning]
If we allow axioms of the form $\Gamma, A \vdash A, \Delta$ then it is
possible to remove the thinning rule as it is redundant and is no longer of
any use.

\end{proposition}

\medskip

Removing the thinning rule gives us a proof system that is essentially the
same as that by Kleene \cite{SK67}. This proof system has the advantage that it
is invertible i.e., if a sequent below the line of an inference is valid then
the sequents above the line are also valid.

\section{Smullyan's Semantic Tableaux}
The Analytic tableaux form a refutation proof system. You begin a proof by
assuming your propositional formula is false. Then a tree is constructed using
the rules. The formula is de-constructed into its constituent sub-formulae
using the rules \textit{And}, \textit{Not-Or} and \textit{Not-Impl}. Case
distinction on the formula is performed by the rules \textit{Or},
\textit{Not-And} and \textit{Impl}. The application of a case distinction rule
causes a branch to occur in the proof tree. If each branch of the tree contains a contradiction then the
formula is refuted.


\label{def:tableaux}
\begin{mydef}[Semantic Tableaux] \hspace*{\fill} \\
\begin{center}

\AxiomC{$A \wedge B$}
\LeftLabel{$(And)$}
\UnaryInfC{$A$}
\alwaysNoLine
\UnaryInfC{$B$}
\DisplayProof \
\AxiomC{$\neg(A \vee B)$}
\LeftLabel{$(Not\hyp{}Or)$}
\UnaryInfC{$\neg A$}
\alwaysNoLine
\UnaryInfC{$\neg B$}
\DisplayProof


\end{center}

\begin{center}

\AxiomC{$A \vee B$}
\LeftLabel{$(Or)$}
\UnaryInfC{$A | B$}
\DisplayProof \
\AxiomC{$\neg(A \wedge B)$}
\LeftLabel{$(Not \hyp{} And)$}
\UnaryInfC{$\neg A | \neg B$}
\DisplayProof


\end{center}

\begin{center}

\AxiomC{$A \to B$}
\LeftLabel{$(Impl)$}
\UnaryInfC{$\neg A | B$}
\DisplayProof \
\AxiomC{$\neg(A \to B)$}
\LeftLabel{$(Not \hyp{} Impl)$}
\UnaryInfC{$A$}
\alwaysNoLine
\UnaryInfC{$\neg B$}
\DisplayProof



\end{center}

\begin{center}

\AxiomC{$\neg \neg A$}
\LeftLabel{$(Not \hyp{} Not)$}
\UnaryInfC{$A$}
\DisplayProof \

\end{center}


\end{mydef}

\medskip

\section{Propagation Rules for St{\aa}lmarck's Tautology Checker}

In the following the proper rules used in St{\aa}lmarck's tautology checking
algorithm are presented from \cite{JN01}.

\label{sec:stalmarck}

\medskip

\begin{mydef}[Formula Equivalence Rules] \hspace*{\fill} \\
\begin{equation}
\AxiomC{}
\UnaryInfC{$P \equiv P$}
\DisplayProof \
\end{equation}

\begin{equation}
\AxiomC{$P \equiv Q$}
\UnaryInfC{$Q \equiv P$}
\DisplayProof \
\end{equation}

\begin{equation}
\AxiomC{$P \equiv Q$}
\AxiomC{$ Q \equiv R  $}
\BinaryInfC{$P \equiv R$}
\DisplayProof \
\end{equation}

\begin{equation}
\AxiomC{$P \equiv \bot$}
\UnaryInfC{$P' \equiv \top$}
\DisplayProof \
\end{equation}

\begin{equation}
\AxiomC{$P \equiv Q$}
\AxiomC{$ Q \equiv \top  $}
\BinaryInfC{$P \equiv \top$}
\DisplayProof \
\end{equation}

\begin{equation}
\AxiomC{$P \equiv Q$}
\AxiomC{$ Q \equiv \bot  $}
\BinaryInfC{$P \equiv \bot $}
\DisplayProof \
\end{equation}


\begin{equation}
\AxiomC{$P \equiv Q$}
\UnaryInfC{$P' \equiv Q'$}
\DisplayProof \
\end{equation}

\begin{equation}
\AxiomC{$P \equiv \top$}
\UnaryInfC{$P' \equiv \bot$}
\DisplayProof \
\end{equation}


\begin{equation}
\AxiomC{$P \equiv \top$}
\AxiomC{$Q \equiv \top$}
\BinaryInfC{$P \equiv Q$}
\DisplayProof \
\end{equation}


\begin{equation}
\AxiomC{$P \equiv \bot$}
\AxiomC{$Q \equiv \bot$}
\BinaryInfC{$P \equiv Q$}
\DisplayProof \
\end{equation}

\begin{equation}
\AxiomC{$P \equiv P'$}
\UnaryInfC{$\bot$}
\DisplayProof \
\end{equation}


\end{mydef}



\begin{mydef}[Propagation Rules]

Rules for conjunction

\begin{equation}
\AxiomC{$P \wedge Q \equiv \top$}
\UnaryInfC{$P \equiv \top$}
\DisplayProof \hspace*{30pt}
\AxiomC{$P \wedge Q \equiv \top$}
\UnaryInfC{$Q \equiv \top$}
\DisplayProof \
\end{equation}


\begin{equation}
\AxiomC{$P \wedge Q \equiv P'$}
\UnaryInfC{$P \equiv \top$}
\DisplayProof \hspace*{30pt}
\AxiomC{$P \wedge Q \equiv Q'$}
\UnaryInfC{$Q \equiv \top$}
\DisplayProof \
\end{equation}



\begin{equation}
\AxiomC{$P \wedge Q \equiv P'$}
\UnaryInfC{$Q \equiv \bot$}
\DisplayProof \hspace*{30pt}
\AxiomC{$P \wedge Q \equiv Q'$}
\UnaryInfC{$P \equiv \bot$}
\DisplayProof \
\end{equation}


\begin{equation}
\AxiomC{$P \equiv \top$}
\UnaryInfC{$P \wedge Q \equiv Q$}
\DisplayProof \hspace*{30pt}
\AxiomC{$Q \equiv \top$}
\UnaryInfC{$P  \wedge Q \equiv P$}
\DisplayProof \
\end{equation}

\begin{equation}
\AxiomC{$P \equiv \bot$}
\UnaryInfC{$P \wedge Q \equiv \bot$}
\DisplayProof \hspace*{30pt}
\AxiomC{$Q \equiv \bot$}
\UnaryInfC{$P  \wedge Q \equiv \bot$}
\DisplayProof \
\end{equation}

\begin{equation}
\AxiomC{$P \equiv Q$}
\UnaryInfC{$P \wedge Q \equiv P$}
\DisplayProof \hspace*{30pt}
\AxiomC{$P \equiv Q$}
\UnaryInfC{$P  \wedge Q \equiv Q$}
\DisplayProof \
\end{equation}

\begin{equation}
\AxiomC{$P \equiv Q'$}
\UnaryInfC{$P  \wedge Q \equiv \bot$}
\DisplayProof \
\end{equation}


Rules for disjunction


\begin{equation}
\AxiomC{$P \vee Q \equiv \bot$}
\UnaryInfC{$P \equiv \bot$}
\DisplayProof \hspace*{30pt}
\AxiomC{$P \vee Q \equiv \bot$}
\UnaryInfC{$ Q \equiv \bot$}
\DisplayProof \
\end{equation}

\begin{equation}
\AxiomC{$P \vee Q \equiv P'$}
\UnaryInfC{$P \equiv \bot$}
\DisplayProof \hspace*{30pt}
\AxiomC{$P \vee Q \equiv Q'$}
\UnaryInfC{$ Q \equiv \bot$}
\DisplayProof \
\end{equation}

\begin{equation}
\AxiomC{$P \vee Q \equiv P'$}
\UnaryInfC{$Q \equiv \top$}
\DisplayProof \hspace*{30pt}
\AxiomC{$P \vee Q \equiv Q'$}
\UnaryInfC{$ P \equiv \top$}
\DisplayProof \
\end{equation}

\begin{equation}
\AxiomC{$P \equiv \top$}
\UnaryInfC{$P \vee Q \equiv \top$}
\DisplayProof \hspace*{30pt}
\AxiomC{$Q \equiv \top$}
\UnaryInfC{$P \vee Q \equiv \top$}
\DisplayProof \
\end{equation}

\begin{equation}
\AxiomC{$P \equiv Q$}
\UnaryInfC{$P \vee Q \equiv P$}
\DisplayProof \hspace*{30pt}
\AxiomC{$P \equiv Q$}
\UnaryInfC{$P \vee Q \equiv Q$}
\DisplayProof \
\end{equation}

\begin{equation}
\AxiomC{$P \equiv Q'$}
\UnaryInfC{$P \vee Q \equiv \top$}
\DisplayProof \
\end{equation}

Rules for implication

\begin{equation}
\AxiomC{$P \to Q \equiv \bot$}
\UnaryInfC{$P  \equiv \top$}
\DisplayProof \
\end{equation}

\begin{equation}
\AxiomC{$P \to Q \equiv \bot$}
\UnaryInfC{$Q \equiv \bot$}
\DisplayProof \
\end{equation}

\begin{equation}
\AxiomC{$P \to Q \equiv P$}
\UnaryInfC{$P \equiv \top$}
\DisplayProof \
\end{equation}

\begin{equation}
\AxiomC{$P \to Q \equiv P$}
\UnaryInfC{$Q \equiv \top$}
\DisplayProof \
\end{equation}

\begin{equation}
\AxiomC{$P \to Q \equiv Q'$}
\UnaryInfC{$P \equiv \bot$}
\DisplayProof \
\end{equation}

\begin{equation}
\AxiomC{$P \to Q \equiv Q'$}
\UnaryInfC{$Q \equiv \bot$}
\DisplayProof \
\end{equation}

\begin{equation}
\AxiomC{$P \equiv \top$}
\UnaryInfC{$P \to Q \equiv Q$}
\DisplayProof \
\end{equation}

\begin{equation}
\AxiomC{$Q \equiv \top$}
\UnaryInfC{$P \to Q \equiv \top$}
\DisplayProof \
\end{equation}

\begin{equation}
\AxiomC{$P \equiv \bot$}
\UnaryInfC{$P \to Q \equiv \top$}
\DisplayProof \
\end{equation}

\begin{equation}
\AxiomC{$Q \equiv \bot$}
\UnaryInfC{$P \to Q \equiv P'$}
\DisplayProof \
\end{equation}

\begin{equation}
\AxiomC{$P \equiv Q'$}
\UnaryInfC{$P \to Q \equiv P'$}
\DisplayProof \hspace*{30pt}
\AxiomC{$P \equiv Q'$}
\UnaryInfC{$P \to Q \equiv Q$}
\DisplayProof \
\end{equation}

Rules for bi-implication

\begin{equation}
\AxiomC{$P \leftrightarrow Q \equiv \top$}
\UnaryInfC{$P \equiv Q$}
\DisplayProof \
\end{equation}

\begin{equation}
\AxiomC{$P \leftrightarrow Q \equiv \bot$}
\UnaryInfC{$P \equiv Q'$}
\DisplayProof \
\end{equation}

\begin{equation}
\AxiomC{$P \leftrightarrow Q \equiv P$}
\UnaryInfC{$Q \equiv \top$}
\DisplayProof \hspace*{30pt}
\AxiomC{$P \leftrightarrow  Q \equiv Q$}
\UnaryInfC{$P \equiv \top$}
\DisplayProof \
\end{equation}

\begin{equation}
\AxiomC{$P \leftrightarrow Q \equiv P'$}
\UnaryInfC{$Q \equiv \bot$}
\DisplayProof \hspace*{30pt}
\AxiomC{$P \leftrightarrow  Q \equiv Q'$}
\UnaryInfC{$P \equiv \bot$}
\DisplayProof \
\end{equation}

\begin{equation}
\AxiomC{$Q \equiv \top$}
\UnaryInfC{$P \leftrightarrow Q \equiv P$}
\DisplayProof \hspace*{30pt}
\AxiomC{$P \equiv \top$}
\UnaryInfC{$P \leftrightarrow  Q \equiv Q$}
\DisplayProof \
\end{equation}


\begin{equation}
\AxiomC{$Q \equiv \bot$}
\UnaryInfC{$P \leftrightarrow Q \equiv P'$}
\DisplayProof \hspace*{30pt}
\AxiomC{$P \equiv \bot$}
\UnaryInfC{$P \leftrightarrow  Q \equiv Q'$}
\DisplayProof \
\end{equation}


\begin{equation}
\AxiomC{$P \equiv Q$}
\UnaryInfC{$P \leftrightarrow Q \equiv \top$}
\DisplayProof \
\end{equation}


\begin{equation}
\AxiomC{$P \equiv Q'$}
\UnaryInfC{$P \leftrightarrow Q \equiv \bot$}
\DisplayProof \
\end{equation}

\end{mydef}

\medskip







\thesischapter{Concrete Railway Model} 


In this chapter we will present the work produced in attempt to provide a
concrete model of the railway (See Chapter 5).

\section{Railway Components}

In the following we have tried to model components of the railway with the intention
that they could be verified individually and then recombined into different
configurations.



 Track Segment 1:
 Node was used to model a straight piece of track. We have simplified the
 topological aspect somewhat
 since the track is generally set up for trains travelling in one direction. We
 have assumed that a train will not travel the wrong way down a track.

\begin{verbatim}

node Track_Segment1(TrainIn, RedLight, GreenLight, WhiteLight : bool)
returns(TrackOccupied, TrainOut, Crash: bool)
var



let

automaton
	initial state EMPTY

	let
	
			TrackOccupied = false; 
			TrainOut = false; 
			Crash = false;
	
	tel
	until
	if (TrainIn) restart OCCUPIED;
	
	state OCCUPIED
	let
		TrackOccupied = true ;
		TrainOut = false;
		Crash = false;
	 
	
	tel
	until
	if ((GreenLight or WhiteLight) and not RedLight) restart TRAINLEAVE;
	if (TrainIn) restart CRASH;
	
	state TRAINLEAVE
	let
	
		TrackOccupied = true;
		TrainOut = true;
		Crash = false;
	
	tel
	until
	if (TrainIn) restart CRASH;
	if (TrainOut) restart EMPTY;
	
	state CRASH
	let
	
		TrackOccupied = true;
		Crash = true;
		TrainOut = false;
	
	tel
	
	returns .. ;

tel

\end{verbatim}

 Track Segment 2:
 Originally we had modelled certain junctions using this component and
 of track using this component. 
 we decided however that we would like to capture all possible ways a train can move across a junction
 we therefore converted all track segments in junctions to using Track Segment 3.

\begin{verbatim}

node Track_Segment2(TrainIn1, TrainIn2, RedLight, GreenLight,
                    WhiteLight, PointsNorm, PointsRev : bool)
returns(TrackOccupied, TrainOut1, TrainOut2, Crash: bool)
var


let


automaton
	initial state EMPTY
	let
	
			TrackOccupied = false; 
			TrainOut1 = false; 
			TrainOut2 = false;
			Crash = false;
	
	tel
	until
	if (TrainIn1 or TrainIn2) restart OCCUPIED;
	
	state OCCUPIED
	let
		TrackOccupied = true;
		TrainOut1 = false;
		TrainOut2 = false;
		Crash = false;
		
	
	tel
	until
	if ((GreenLight or WhiteLight) and not RedLight) restart TRAINLEAVE;
	if (TrainIn1 or TrainIn2) restart CRASH;
	
	state TRAINLEAVE
	let
	
		TrackOccupied = true;
		TrainOut1 =  if (PointsRev) then true else false;
		TrainOut2 = if (PointsNorm) then true else false;
		Crash = false;
	
	tel
	until 
	if (TrainIn1 or TrainIn2) restart CRASH;
	if (TrainOut1 or TrainOut2) restart EMPTY;
	
	state CRASH
	let
	
		TrackOccupied = true;
		Crash = true;
		TrainOut1 = false;
		TrainOut2 = false;
	tel
	
	returns .. ;

tel

\end{verbatim}

 Track Segment 3: This is the node that was used to model junctions in the track. The just has
 been modelled in such a way that the direction of
 travel along with the points dictate how a train exits the track. 




\begin{verbatim}


node Track_Segment3(TrainIn1, TrainIn2, TrainIn3, RedLight, 
GreenLight, WhiteLight, PointsNorm , PointsRev : bool)
returns(TrackOccupied, TrainOut1, TrainOut2, TrainOut3 , Crash: bool)
var

Direction : bool;

let

Direction = false -> if (TrainIn1) then true else (if (TrainIn3) then false 
							else pre Direction); 

automaton
	initial state EMPTY
	let
	
			TrackOccupied = false; 
			TrainOut1 = false; 
			TrainOut2 = false;
			TrainOut3 = false;
			
			Crash = false;
	
	tel
	until
	if (TrainIn1 or TrainIn2 or TrainIn3) restart OCCUPIED;
	
	state OCCUPIED
	let
		TrackOccupied = true ;
		TrainOut1 = false;
		Crash = false;
		TrainOut2 = false;
		TrainOut3 = false;
	 
	
	tel
	until
	if ((GreenLight or WhiteLight) and not RedLight) restart TRAINLEAVE;
	if (TrainIn1 or TrainIn2 or TrainIn3) restart CRASH;
	
	state TRAINLEAVE
	let
	
		TrackOccupied = true;
		TrainOut1 = if (Direction) then false else true ;
		Crash = false;
		TrainOut2 = if (Direction and PointsRev) then true else false ;
		TrainOut3 = if (Direction and PointsNorm) then true else false;
	
	tel
	until
	if (TrainIn1 or TrainIn2 or TrainIn3) restart CRASH;
	if (TrainOut1 or TrainOut2 or TrainOut3) restart EMPTY;
	
	state CRASH
	let
	
		TrackOccupied = true;
		Crash = true;
		TrainOut1 = false;
		TrainOut2 = false;
		TrainOut3 = false;
	
	tel
	
	returns .. ;

tel


\end{verbatim}



\begin{verbatim}

node Route(RouteCall, RouteSet, PointsLocked,
 LightsSet, W_track_clear, G_track_clear: bool)
returns(RouteSelected, DrivePL, DriveG, DriveW, DriveR: bool) 
let
automaton
	initial state STATE1
	unless
	if (RouteCall) restart STATE2;
	let
	
		RouteSelected = false;
		DrivePL = false;
		DriveG = false;
		DriveW = false;
		DriveR = true;
	
	tel
	state STATE2
	unless
	if (not RouteSet) restart STATE1;
	if (RouteCall and PointsLocked and LightsSet) restart STATE3;
	let
	
	
		RouteSelected = false; 
		DrivePL = true;
		DriveG = if (W_track_clear and G_track_clear)
						then true
						else false;
		DriveW = if (W_track_clear and not G_track_clear)	
						then true
						else false;
		DriveR = if ( (W_track_clear and G_track_clear) 
                      or (W_track_clear and not G_track_clear))
						then false
						else true;
						
	
	tel
	state STATE3
	unless
	if (not RouteSet) restart STATE1;
	let
	
		RouteSelected = true;
		DrivePL = true;
		DriveG = if (W_track_clear and G_track_clear)
						then true
						else false;
		DriveW = if (W_track_clear and not G_track_clear)	
						then true
						else false;
		DriveR = if ( (W_track_clear and G_track_clear) 
                      or (W_track_clear and not G_track_clear))
						then false
						else true;
	tel
	returns .. ;
	

tel

\end{verbatim}

The following is the \scade \ node that was used to model a point. It contains a
finite state machine that models the 4 possible states a point can be
in: Normal and free , Normal and locked, Reverse and free, Reverse and
locked. One further state that could be added in future is the Unknown
state. This is where the point is in neither Reverse or Normal but in some
indeterminate state. 

\begin{verbatim}

node Point(Normal, Reverse, Occupied : bool)
returns(NLock, NFree, RLock, RFree: bool)
let

automaton
  initial state NORMAL_FREE
	unless
	if (Occupied) restart NORMAL_LOCK;

	if (Reverse and not Normal and not Occupied) restart REVERSE_FREE;
	let
		NLock = false;
		RLock = false;
		NFree = true ;
		RFree = false ;
	tel
	until

	
	state NORMAL_LOCK
	unless
	if (not Occupied) restart NORMAL_FREE;
	let
		NLock = true;
		RLock = false;
		NFree = false ;
		RFree = false ;
	
	tel
	
	state REVERSE_FREE
	unless
	if (Occupied) restart REVERSE_LOCK;
	if (not Reverse and Normal and not Occupied) restart NORMAL_FREE;
	let
	
		NLock = false;
		RLock = false;
		NFree = false;
		RFree = true ;
	
	
	
	tel


	
	state REVERSE_LOCK
	unless
	if (not Occupied) restart REVERSE_FREE;
	let
		NLock = false;
		RLock = true;
		NFree = false;
		RFree = false;
			
			
	tel
	
	returns .. ;

tel

\end{verbatim}

Pointif is a model of a point using if statements instead of the finite state
machines.


\begin{verbatim}

node Pointif(Normal, Reverse, Occupied : bool)
returns(NLock, NFree, RLock, RFree: bool)
let


		NLock =  (Occupied) ->
								 ( (pre NFree and Occupied) or (pre NLock and Occupied));
								 
		RLock = false ->  (pre RFree or pre RLock) and Occupied ;
		NFree = ( (Normal and not Reverse) or (not Normal and not Reverse)
                                  or (Normal and Reverse)) and not Occupied
					 -> ( ((pre RFree and not Reverse and Normal) or (pre NFree and
					 (not Reverse or (Reverse and Normal))or pre NLock)) and not Occupied);
		
		RFree =  (not Occupied and (Reverse and not Normal)) -> 
					 ( (pre NFree and not Normal and Reverse and not Occupied ) or  
					( (pre RFree and (not Normal and (notOccupied or not Reverse)
                                or (Normal and Reverse)) )  and not Occupied) 
					or (pre RLock and not Occupied)) ;


tel

\end{verbatim}



\begin{verbatim}

node PointEquiv(Normal, Reverse, Occupied: bool)
returns(Equivalent , NLock1, NFree1, RLock1, RFree1, 
              NLock2, NFree2, RLock2, RFree2 : bool)



let


	NLock1, NFree1, RLock1, RFree1	= 	Point(Normal, Reverse, Occupied);
	NLock2, NFree2, RLock2, RFree2	= 	Pointif(Normal, Reverse, Occupied);
	
	Equivalent =((NLock1 and NLock2) or (not NLock1 and not NLock2)) and
				((NFree1 and NFree2) or (not NFree1 and not NFree2)) and
				((RFree1 and RFree2) or (not RFree1 and not RFree2)) and
				((RLock1 and RLock2) or (not RLock1 and not RLock2));

tel

\end{verbatim}


\begin{verbatim}

node Point2(Normal, Reverse, Occupied : bool)
returns(NLock, NFree, RLock, RFree: bool)
let

automaton
  initial state INITIAL
  unless
  	if (false -> Occupied) restart NORMAL_LOCK;
	if (false -> Normal) restart NORMAL_FREE;
	if (false -> Reverse) restart REVERSE_FREE;
	if (false -> not Reverse and not  Normal and not Occupied) restart NORMAL_FREE;
	let
	
		NLock = false;
		RLock = false;
		NFree = true ;
		RFree = false ;
	
	tel
	until		

	
	state NORMAL_FREE
	unless
	if (Occupied) restart NORMAL_LOCK;

	if (Reverse and not Normal and not Occupied) restart REVERSE_FREE;
	let
		NLock = false;
		RLock = false;
		NFree = true ;
		RFree = false ;
	tel
	until

	
	state NORMAL_LOCK
	unless
	if (not Occupied) restart NORMAL_FREE;
	let
		NLock = true;
		RLock = false;
		NFree = false ;
		RFree = false ;
	
	tel
	
	state REVERSE_FREE
	unless
	if (Occupied) restart REVERSE_LOCK;
	if (not Reverse and Normal and not Occupied) restart NORMAL_FREE;
	let
	
		NLock = false;
		RLock = false;
		NFree = false;
		RFree = true ;
	
	
	
	tel


	
	state REVERSE_LOCK
	unless
	if (not Occupied) restart REVERSE_FREE;
	let
		NLock = false;
		RLock = true;
		NFree = false;
		RFree = false;
			
			
	tel
	
	returns .. ;
tel

\end{verbatim}

\begin{verbatim}

node Pointif2(Normal, Reverse, Occupied : bool)
returns(NLock, NFree, RLock, RFree: bool)
let


		NLock =  false ->
								 ( (pre NFree and Occupied) or (pre NLock and Occupied));
								 
		RLock = false ->  (pre RFree or pre RLock) and Occupied ;
		NFree = true
					 -> ( ((pre RFree and not Reverse and Normal) or (pre NFree and
					 (not Reverse or (Reverse and Normal))or pre NLock)) and not Occupied);
		
		RFree =  false -> 
					 ( (pre NFree and not Normal and Reverse and not Occupied ) or  
					( (pre RFree and (not Normal and
(not Occupied or not Reverse) or (Normal and Reverse)) )  and not Occupied) 
					or (pre RLock and not Occupied)) ;
tel

\end{verbatim}

\begin{verbatim}

node PointEquiv2(Normal, Reverse, Occupied: bool)
returns(Equivalent, Equivalent2, Equivalent3, NLock1,
 NFree1, RLock1, RFree1, NLock2, NFree2,
 RLock2, RFree2, NLock3, NFree3, RLock3, RFree3,
	NLock4, NFree4, RLock4, RFree4 : bool)
let


	NLock1, NFree1, RLock1, RFree1	= 	Point2(Normal, Reverse, Occupied);
	NLock2, NFree2, RLock2, RFree2	= 	Pointif2(Normal, Reverse, Occupied);
	NLock3, NFree3, RLock3, RFree3  = 	Point(Normal, Reverse, Occupied);
	NLock4, NFree4, RLock4, RFree4 	= 	Pointif(Normal, Reverse, Occupied);
	
	
	
	
	Equivalent =((NLock1 and NLock2) or (not NLock1 and not NLock2)) and
			((NFree1 and NFree2) or (not NFree1 and not NFree2)) and
			((RFree1 and RFree2) or (not RFree1 and not RFree2)) and
			((RLock1 and RLock2) or (not RLock1 and not RLock2));
	
	Equivalent2 = true -> ((NLock1 and NLock3) or (not NLock1 and not NLock3)) and
			((NFree1 and NFree3) or (not NFree1 and not NFree3)) and
			((RFree1 and RFree3) or (not RFree1 and not RFree3)) and
			((RLock1 and RLock3) or (not RLock1 and not RLock3));
	
	Equivalent3 = true -> ((NLock2 and NLock4) or (not NLock2 and not NLock4)) and
			((NFree2 and NFree4) or (not NFree2 and not NFree4)) and
			((RFree2 and RFree4) or (not RFree2 and not RFree4)) and
			((RLock2 and RLock4) or (not RLock2 and not RLock4));
		
	
tel

\end{verbatim}

\section{Signals}
Signals were modelled using the finite state machines in \scade. The signals
and the aspects they contain were modelled separately. A signal can
be though of as a device which controls the aspects it contains.  They have an
Initial state in which the the red aspect is driven. Subsequent states depend
on the value of the inputs. 

\begin{verbatim}

node Light3Aspect(Red, White , Green: bool)
returns(LightsSet, R_O_D , W_O_D, G_O_D
		
				: bool)
var
G_O_A,  G_O_R, W_O_A, W_O_R, R_O_A, R_O_R,
G_I_A, G_I_D, G_I_R, W_I_A, W_I_D, W_I_R, R_I_A, R_I_D,  R_I_R, 
G_S_A, G_S_D, G_S_R, W_S_A, W_S_D, W_S_R, R_S_A, R_S_D,  R_S_R : bool;

let 

	G_S_A = false -> (pre G_I_A);
	G_S_D = false -> (pre G_I_D);
	G_S_R = false -> (pre G_I_R);
	W_S_A = false -> (pre W_I_A);
	W_S_D = false -> (pre W_I_D);
	W_S_R = false -> (pre W_I_R);
	R_S_A = false -> (pre R_I_A);
	R_S_D = false -> (pre R_I_D);
	R_S_R = false -> (pre R_I_R);

	automaton
	initial state INITIALISE
	let
			G_I_A = false;
			G_I_D = false;
			G_I_R = false;
			W_I_A = false;
			W_I_D = false;  
			W_I_R = false;
			R_I_A = true;
			R_I_D = true;
			R_I_R = false;
			LightsSet = false;

	tel
	until
	if (R_O_A and R_O_D and Red) resume RED;
	if (R_O_A and R_O_D and White) resume WHITE;
	if (R_O_A and R_O_D and Green) resume GREEN;
	state RED

	let
		automaton
		initial state SETAVAIL
		unless
		if (R_S_A and not R_S_D) resume TRANSITIONSTATE; 
			let
			G_I_A = G_S_A;
			G_I_D = G_S_D;
			G_I_R = G_S_R;
			W_I_A = W_S_A;
			W_I_D = W_S_D; 
			W_I_R = W_S_R;	
			R_I_A = true;
			R_I_D = R_S_D;
			R_I_R = R_S_R;
			LightsSet = false;
			tel
			
		state TRANSITIONSTATE
		unless
		if (R_S_A and R_S_D) resume LIGHTSSET;
			let
			G_I_A = G_S_A;
			G_I_D = false;
			G_I_R = G_S_R;
			W_I_A = W_S_A;
			W_I_D = false; 
			W_I_R = W_S_R;	
			R_I_A = true;
			R_I_D = true;
			R_I_R = R_S_R;
			LightsSet = false;
			
			
			tel
		state LIGHTSSET
		let
					G_I_A = false;
					G_I_D = false;
					G_I_R = G_S_R;
					W_I_A = false;
					W_I_D = false; 
					W_I_R = W_S_R;	
					R_I_A = true;
					R_I_D = true;
					R_I_R = R_S_R;
					LightsSet = true;
		tel
		
		returns .. ;

	tel
	until 
	if (Green and LightsSet) resume GREEN;
	if (White and LightsSet) resume WHITE;
	
	state WHITE
	let

		automaton
		initial state SETAVAIL
		unless
		if (W_S_A and not W_S_D) resume TRANSITIONSTATE; 
		
			let
			G_I_A = G_S_A;
			G_I_D = G_S_D;
			G_I_R = G_S_R;
			W_I_A = true;
			W_I_D = G_S_D; 
			W_I_R = W_S_R;	
			R_I_A = R_S_A;
			R_I_D = R_S_D;
			R_I_R = R_S_R;
			LightsSet = false;
			tel
		state TRANSITIONSTATE
		unless
		if (R_S_A and R_S_D) resume LIGHTSSET;
			let
			G_I_A = G_S_A;
			G_I_D = false;
			G_I_R = G_S_R;
			W_I_A = true;
			W_I_D = true; 
			W_I_R = W_S_R;	
			R_I_A = R_S_A;
			R_I_D = false;
			R_I_R = R_S_R;
			LightsSet = false;
			
			
			tel
		state LIGHTSSET
		let
					G_I_A = false;
					G_I_D = false;
					G_I_R = G_S_R;
					W_I_A = true;
					W_I_D = true; 
					W_I_R = W_S_R;	
					R_I_A = false;
					R_I_D = false;
					R_I_R = R_S_R;
					LightsSet = true;
		tel
		
		returns .. ;

	tel
	until
	if ((Red and LightsSet)or (not Red and not White
            and not Green and LightsSet)) resume RED;
	if (Green and LightsSet) resume GREEN;
	
	state GREEN
	let
	
		automaton
		initial state SETAVAIL
		unless
		if (R_S_A and not R_S_D) resume TRANSITIONSTATE; 
			let
			G_I_A = true;
			G_I_D = G_S_D;
			G_I_R = G_S_R;
			W_I_A = W_S_A;
			W_I_D = W_S_D;
			W_I_R = W_S_R;	
			R_I_A = R_S_A;
			R_I_D = R_S_D;
			R_I_R = R_S_R;
			LightsSet = false;
			tel
			
		state TRANSITIONSTATE
		unless
		if (R_S_A and R_S_D) resume LIGHTSSET;
			let
			G_I_A = true;
			G_I_D = true;
			G_I_R = G_S_R;
			W_I_A = W_S_A;
			W_I_D = false; 
			W_I_R = W_S_R;	
			R_I_A = R_S_A;
			R_I_D = false;
			R_I_R = R_S_R;
			LightsSet = false;
			
			
			tel
		state LIGHTSSET
		let
					G_I_A = true;
					G_I_D = true;
					G_I_R = G_S_R;
					W_I_A = false;
					W_I_D = false; 
					W_I_R = W_S_R;	
					R_I_A = false;
					R_I_D = false;
					R_I_R = R_S_R;
					LightsSet = true;
		tel
		
		returns .. ;

				
		
	tel
	until
	if ((Red and LightsSet) or 
          (not Red and not White and not Green and LightsSet )) resume RED;
	if (White and LightsSet) resume WHITE;
	returns .. ;
	
	G_O_A, G_O_D, G_O_R = SignalAspect(G_I_A , G_I_D , G_I_R);
	W_O_A, W_O_D, W_O_R = SignalAspect(W_I_A , W_I_D , W_I_R);
	R_O_A, R_O_D, R_O_R = SignalAspect(R_I_A , R_I_D , R_I_R);

	
	-- Safety Conditions for a 3 Aspect Signal
	-- ConLight = not (G_O_D and W_O_D  or
       -- G_O_D and R_O_D or W_O_D and R_O_D);
	--  Onelight =  G_O_D or W_O_D or R_O_D;
	
tel

\end{verbatim}

\begin{verbatim}

node Light2Aspect(Red,Green: bool)
returns(LightsSet, R_O_D, G_O_D
		
				: bool)
var
G_O_A, G_O_R, R_O_A, R_O_R,
G_I_A, G_I_D, G_I_R, R_I_A, R_I_D,  R_I_R,
G_S_A, G_S_D, G_S_R, R_S_A, R_S_D,  R_S_R : bool;

let 


	G_S_A = false -> (pre G_I_A);
	G_S_D = false -> (pre G_I_D);
	G_S_R = false -> (pre G_I_R);

	R_S_A = false -> (pre R_I_A);
	R_S_D = false -> (pre R_I_D);
	R_S_R = false -> (pre R_I_R);

	automaton
	initial state INITIALISE
	let
			G_I_A = false;
			G_I_D = false;
			G_I_R = false;
			R_I_A = true;
			R_I_D = true;
			R_I_R = false;
			LightsSet = false;

	tel
	until
	if (R_O_A and R_O_D and Red) resume RED;
	if (R_O_A and R_O_D and Green) resume GREEN;
	state RED

	let
		automaton
		initial state SETAVAIL
		unless
		if (R_S_A and not R_S_D) resume TRANSITIONSTATE; 
			let
			G_I_A = G_S_A;
			G_I_D = G_S_D;
			G_I_R = G_S_R;
			R_I_A = true;
			R_I_D = R_S_D;
			R_I_R = R_S_R;
			LightsSet = false;
			tel
			
		state TRANSITIONSTATE
		unless
		if (R_S_A and R_S_D) resume LIGHTSSET;
			let
			G_I_A = G_S_A;
			G_I_D = false;
			G_I_R = G_S_R;
			R_I_A = true;
			R_I_D = true;
			R_I_R = R_S_R;
			LightsSet = false;
			
			
			tel
		state LIGHTSSET
		let
					G_I_A = false;
					G_I_D = false;
					G_I_R = G_S_R;
					R_I_A = true;
					R_I_D = true;
					R_I_R = R_S_R;
					LightsSet = true;
		tel
		
		returns .. ;

	tel
	until 
	if (Green and LightsSet) resume GREEN;
	state GREEN
	let
	
		automaton
		initial state SETAVAIL
		unless
		if (R_S_A and not R_S_D) resume TRANSITIONSTATE; 
			let
			G_I_A = true;
			G_I_D = G_S_D;
			G_I_R = G_S_R;
			R_I_A = R_S_A;
			R_I_D = R_S_D;
			R_I_R = R_S_R;
			LightsSet = false;
			tel
			
		state TRANSITIONSTATE
		unless
		if (R_S_A and R_S_D) resume LIGHTSSET;
			let
			G_I_A = true;
			G_I_D = true;
			G_I_R = G_S_R;
			R_I_A = R_S_A;
			R_I_D = false;
			R_I_R = R_S_R;
			LightsSet = false;
			
			
			tel
		state LIGHTSSET
		let
					G_I_A = true;
					G_I_D = true;
					G_I_R = G_S_R;
					R_I_A = false;
					R_I_D = false;
					R_I_R = R_S_R;
					LightsSet = true;
		tel
		
		returns .. ;

				
		
	tel
	until
	if ((Red and LightsSet) or 
          (not Red and not Green and LightsSet )) resume RED;
	returns .. ;
	
	G_O_A, G_O_D, G_O_R = SignalAspect(G_I_A , G_I_D , G_I_R);
	R_O_A, R_O_D, R_O_R = SignalAspect(R_I_A , R_I_D , R_I_R);

	--  ConLight = not (G_O_D and R_O_D);
	--   Onelight =  G_O_D or R_O_D;
\end{verbatim}	

Currently the fixed red constantly outputs a boolean stream containing the
value true. Further behaviour could be modelled at a later stage such as
failure and reporting.

\begin{verbatim}
node FixedRed()
returns(Red : bool)
var

let

Red = true; 
	

tel
\end{verbatim}

The following is the \scade \ model for a signal aspect.

\begin{verbatim}

node SignalAspect(a,d,r : bool) returns (Avail, Driven, Report: bool)
let


	
	automaton
		initial state STATE_1
		unless 
		if (not a) restart STATE_5;
		if (d) restart STATE_2;
		if (r) restart STATE_4;
		let
			
			Avail = true;
			Driven = false;
			Report = false;
			
		tel
	
		state STATE_2
		unless
		if (not d) restart STATE_1;
		if (not a) restart STATE_6;
		if (r) restart STATE_3;
		let
			
			Avail = true;
			Driven = true;
			Report = false;
		
		
		tel
		
		
		state STATE_3
		unless
		if (not a) restart STATE_8;
		if (not d) restart STATE_4;
		if (not r) restart STATE_2;
		let
			
			Avail = true;
			Driven = true;
			Report = true;
		
		
		tel
		
		state STATE_4
		unless
		if (not a) restart STATE_7;
		if (d) restart STATE_3;
		if (not r) restart STATE_1;
		let
				
				
			Avail = true;
			Driven = false;
			Report = true;
		
		tel
		
		state STATE_5
		unless 
		if (a) restart STATE_1;
		if (r) restart STATE_7;
		let 
			
			
			Avail = false;
			Driven = false;
			Report = false;
		
		
		
		tel
		
		state STATE_6
		unless
		if (a) restart STATE_2;
		if (r) restart STATE_8;
		let
		
			Avail = false;
			Driven = true;
			Report = false;
		
		tel
		
		state STATE_7
		unless
		if (a) restart STATE_4;
		if (not r) restart STATE_5;
		let
			
			Avail = false;
			Driven = false;
			Report = true;
		
		tel
		
		state STATE_8
		unless
		if (a) restart STATE_3;
		if (not r) restart STATE_6;
		let
			
			
			Avail = false;
			Driven = true;
			Report = true;	
				
				
		tel
		returns .. ;
		
		-- AspectSafe = true -> Avail or pre Avail or 
               (not Avail and not pre Avail and 
               (not Driven  and not pre Driven) or (Driven and pre Driven));
		

tel

\end{verbatim}

\section{Route Controller}

\begin{verbatim}

node RouteController(v1204_1_R, v1204_2_R, v1206_R, v1205_R, v1203_R, v1201_R,
 v1204_1_RS, v1204_2_RS, v1206_RS, v1205_RS, v1203_RS, v1201_RS : bool)
returns (v1204_1_A, v1204_2_A, v1206_A, v1205_A, v1203_A , v1201_A,
		v1204_1_C, v1204_2_C, v1206_C, v1205_C, v1203_C, v1201_C,
		v1204_1_S, v1204_2_S, v1206_S, v1205_S, v1203_S, v1201_S,
		ConNP1, ConNP2, ConNP3, ConNP4 , ConNP5
		: bool)

let


	-- automaton for conflicting routes

	automaton
	initial state INIT
	unless
    if (v1204_1_R) restart ROUTESET2;
	if (v1204_2_R) restart ROUTESET1;
	if (v1206_R) restart ROUTESET1;
	if (v1205_R) restart ROUTESET3;

	let
		-- All routes are available initially
		v1204_1_A = true ;
		v1204_2_A = true ;
		v1206_A = true ; 
		v1205_A = true ;
		
		-- No routes have been called in the intial state
		
		v1204_1_C = false ;
		v1204_2_C = false ;
		v1206_C = false ; 
		v1205_C = false ;
		
		-- No routes are selected in the intial state
		
		v1204_1_S = false ;
		v1204_2_S = false ;
		v1206_S = false ; 
		v1205_S = false ;
		
	tel
	
	
	state ROUTESET1
	unless
	if (not v1206_R  and not v1204_1_R) restart INIT;
	let
		--
		v1204_1_A = if (v1204_2_R) then true else false;
		v1204_2_A = false ;
		v1206_A = if (v1206_R) then true else false; 
		v1205_A = false ;
		
		-- The requested route is called
		v1204_1_C = if (v1204_2_R) then true else false;
		v1204_2_C = false ; 
		v1206_C = if (v1206_R) then true else false; 
		v1205_C = false ;
		
		
		-- if statement for 1206
		
	       v1206_S = if (not v1206_RS) then false else true;
		
		-- if statement for 1204_2
	
		
				
		v1204_1_S = if (not v1204_2_RS) then false else true;		
		
		v1204_2_S = false ; 
		v1205_S = false ;
	
	tel
	
	state ROUTESET2
	unless
	if (not v1204_2_R) restart INIT;
	let
			
		v1204_1_A = false ;
		v1204_2_A = true ;
		v1206_A = false ; 
		v1205_A = false ;
		
		-- The requested route is called
		
		v1204_1_C = false ; 
		v1204_2_C = true ;
		v1206_C = false ; 
		v1205_C = false ;
		
		-- automaton for 1204_1
		
		
		automaton
		initial state NOT_SEL
		unless 
		if (v1204_2_RS) restart SEL;
		let
		
			v1204_2_S = false;
		
		tel
		state SEL
		let
		
			v1204_2_S = true;
		
		
		tel	
		returns .. ;
		
		v1206_S = false;
		v1204_1_S = false;
		v1205_S = false;
		
	
	tel
	
	state ROUTESET3
	unless
	if (not v1205_R) restart INIT;
	let
			
		v1204_1_A = false ;
		v1204_2_A = false ;
		v1206_A = false ; 
		v1205_A = true ;
		
		-- The requested route is called
		
		v1204_1_C = false ; 
		v1204_2_C = false ;
		v1206_C = false ; 
		v1205_C = true ;
		
		-- automaton for 1205
		
		automaton
		initial state NOT_SEL
		unless 
		if (v1205_RS) restart SEL;
		let
		
			v1205_S = false;
		
		tel
		state SEL
		let
		
			v1205_S = true;
		
		
		tel	
		returns .. ;
		
		v1204_1_S = false;
		v1204_2_S = false;
		v1206_S = false;
	
	tel
	returns .. ;
	
	--- Routes with no conflicts are always available
	
	
	
	v1203_A = true;
	v1201_A = true;
	
	-- flows for for route 1203
	
	
	
			
	v1203_C = if (v1203_R) then true else false;
	v1203_S = if (v1203_RS) then true else false;	
	
		
		
	
	-- flows for route 1201
	
		
			
	v1201_C = if(v1201_R) then true else false;
	v1201_S = if (v1201_RS) then true else false;	
	
		
		--Safety Conditions for Conflicting Routes
		
		ConNP1 = not( v1204_1_S and v1205_S); 
		ConNP2 = not( v1204_1_S and v1204_2_S);
		ConNP3 = not( v1204_2_S and v1205_S);
		ConNP4 = not( v1204_2_S and v1206_S);
		ConNP5 = not( v1206_S and v1205_S);
	
	
tel
\end{verbatim}

\section{Railway Segment Model}

The following contains the \scade \ model for concrete railway example. Below
is a list of variable suffixes.
 

\begin{itemize}

\item Track Segments
\begin{itemize}
\item TO : Track Occupied
\item O : Train Out
\item CR : Crash
\end{itemize}

\item Signals
\begin{itemize}
\item RED : Red aspect is showing on the signal
\item GRE : Green aspect is showing on the signal
\item WHI : White aspect is showing on the signal
\item LS  : Lights set
\end{itemize}



\item -- Points
\begin{itemize}
\item NL : Points locked normal
\item RL : Points locked reverse
\item NF : Points free and normal
\item RF : Points Free and reverse
\end{itemize}
\item Routes

\begin{itemize}
\item R: Route Requested
\item RS : Route Selected
\item RC : Route Called
\item S : Route Set
\item A : Available
\item DP : Drive points , caused by a route being called.
\item DR : Drive Red
\item DG :
\item DW :



\item WTC : The track segments required for a white light are clear.
\item GTC : The track segments required for a green light are clear.

\end{itemize}



\end{itemize}

Each track component has a unique identifier.

\begin{center}
    \begin{tabular}{ | l |  p{5cm} |}
    \hline
    Component Type & Component Identifiers \\ \hline
    Track &   1001, 1002, 1003, 1004, 1005, 1006, 1007, 1008, 1009, 1010, 1011 \\ \hline
    Points &  1101, 1102, 1103, 1104  \\ \hline
    Signals & 1201, 1202, 1203, 1204, 1205, 1206, 1207 , 1208, 1209, 1210 \\
    \hline
    \end{tabular}
\end{center}



\begin{verbatim}


node AbstractRailway(v1204_1_R, v1204_2_R, v1206_R , v1205_R,
                           v1203_R, v1201_R, TrainIn : bool)

returns(sPLRS_1206, sPLRS_1204_1, sPLRS_1204_2,
	sPLRS_1205,  sNPA_1206, sNPA_1204, sNPA_1205, sNPATR_1206,
       sNPATR_1204_1, sNPATR_1204_2, sNPATR_1205, nocrash, sTI1001  : bool)
var 
v1205_WTC, v1204_1_WTC, v1204_2_WTC, v1206_WTC, v1203_WTC , v1201_WTC : bool;
v1205_GTC, v1204_1_GTC, v1204_2_GTC, v1206_GTC, v1203_GTC , v1201_GTC : bool;
	
 v1204_1_A, v1204_2_A,
 v1206_A, v1205_A, v1204_1_S, v1204_2_S, v1206_S, v1205_S ,
v1101_NF , v1101_NL , v1101_RF, v1101_RL, v1102_NF , v1102_NL ,
    v1102_RF, v1102_RL ,v1103_NF, v1103_NL, v1103_RF , v1103_RL,
	v1104_NF, v1104_NL, v1104_RF , v1104_RL,
	
	v1204_1_RS, v1204_1_DP, v1204_1_DG , v1204_1_DW , v1204_1_DR, 
	v1204_2_RS , v1204_2_DP, v1204_2_DG , v1204_2_DW , v1204_2_DR, 
	v1206_RS , v1206_DP, v1206_DG, v1206_DW, v1206_DR,
	v1205_RS , v1205_DP, v1205_DG , v1205_DW , v1205_DR,
	v1203_RS , v1203_DP, v1203_DG, v1203_DW, v1203_DR,
	v1201_RS , v1201_DP, v1201_DG, v1201_DW, v1201_DR,
	v1204_2_RC, v1204_LS, v1206_LS, v1205_RC, v1205_LS, v1203_RC, 
	v1203_LS, v1201_RC, v1201_LS,  v1201_RED, v1203_RED, v1204_RED,
	v1205_RED, v1206_RED, v1204_1_RC, v1206_RC, v1203_S,
    v1201_S, v1203_A,  v1201_A, v1201_GRE, v1201_WHI,
	v1001_TO , v1001_O, v1002_TO , v1002_O, v1003_TO, v1003_O,
	v1204_GRE, v1004_TO , v1004_O_2,
	v1006_TO, v1006_O, v1205_GRE,  v1005_TO, v1005_O_2,
	v1007_TO, v1007_O, v1008_TO, v1008_O, v1009_TO, v1009_O,
	v1010_TO, v1010_O_2 , v1010_O_3, v1011_TO, v1011_O,
	v1009_O_2, v1004_O,  v1010_O, v1005_O, v1203_WHI, v1203_GRE, v1206_GRE,
	v8255_1_D, v1208_D, v1209_D, v8257_2_D ,
      v1001_CR, v1002_CR, v1003_CR, v1004_CR, v1011_CR, v1005_CR,
	v1006_CR,  v1010_CR, v1009_CR, v1008_CR, v1007_CR , v1005_O_3, v1004_O_3, v1009_O_3


	: bool;
let

-- East Bound Line


-- Track 1001 
v1001_TO , v1001_O, v1001_CR = 
      Track_Segment1( TrainIn, v1201_RED, v1201_GRE, v1201_WHI );  

-- Track 1002
-- Since there is no light on this segment
-- any train can proceed to the next piece of track
-- hence green and white are set to true untill i find a way of modelling this.

v1002_TO , v1002_O, v1002_CR =
      Track_Segment1(v1001_O , false, true , true);

-- Track 1003
v1003_TO, v1003_O, v1003_CR	=
     Track_Segment1(v1002_O, v1204_RED, v1204_GRE, false);
	
-- Track 1004
v1004_TO, v1004_O , v1004_O_2, v1004_O_3, v1004_CR =
 	Track_Segment3(v1003_O, v1009_O_2 , v1005_O_3,
                    false, true, false, v1101_NL , v1101_RL);
	
-- Track 1005
v1005_TO, v1005_O, v1005_O_2 , v1005_O_3, v1005_CR = 
      Track_Segment3(v1006_O, false, v1004_O_3,
                   false , true, false , v1103_NL , v1103_RL);

-- Track 1006
v1006_TO, v1006_O, v1006_CR = Track_Segment1(v1005_O, v1205_RED, v1205_GRE, false);

-- West Bound Line
-- Track 1007
v1007_TO, v1007_O, v1007_CR = Track_Segment1(v1008_O, false, true, false);
	
-- Track 8002
-- Unclear if this track segment is part of the scheme.
-- v8002_TO, v8002_O, v8002_CR = 
-- Track_Segment1(v1007_O, v1202_RED , v1202_GRE, v1202_WHI);
	
-- Track 1008
v1008_TO, v1008_O, v1008_CR	=
      Track_Segment1(v1009_O_3, v1203_RED, v1203_GRE , v1203_WHI);
	
-- Track 1009
v1009_TO, v1009_O, v1009_O_2 , v1009_O_3, v1009_CR =
       Track_Segment3( v1010_O, v1004_O_2, 
       false ,  false , true, false, v1102_NL , v1102_RL);
	
-- Track 1010
v1010_TO, v1010_O, v1010_O_2 , v1010_O_3, v1010_CR =
       Track_Segment3(v1009_O, v1005_O_2,
       v1011_O , false, true, false, v1104_NL, v1104_RL);
	
-- Track 1011
v1011_TO, v1011_O, v1011_CR	 = Track_Segment1(v1010_O_3, v1206_RED, v1206_GRE, false);
	
	
	-- Points
-- Left Points 1101/2
	
	v1101_NF , v1101_NL , v1101_RF, v1101_RL =	
      Point( (v1204_1_DP or v1205_DP or v1205_DP), v1204_2_DP, v1004_TO );
	
      v1102_NF , v1102_NL , v1102_RF, v1102_RL =
 	Point( (v1204_1_DP or v1205_DP or v1205_DP), v1204_2_DP, v1009_TO );

-- Right Points 1103/4


	v1103_NF, v1103_NL, v1103_RF , v1103_RL =
      Point((v1204_1_DP or v1204_2_DP or v1206_DP), v1205_DP, v1005_TO);
	
      v1104_NF, v1104_NL, v1104_RF , v1104_RL =
      Point((v1204_1_DP or v1204_2_DP or v1206_DP), v1205_DP,  v1010_TO);

	
	
-- Routes
-- Route 1204(1) (Ends 8255)
-- Locks Points Normal 
-- 1101/2 
-- 1103/4
-- Tracks Clear Green : 1004, 1005 , 1006
-- Tracks Clear White

	v1204_1_WTC = false;
	v1204_1_GTC = v1004_TO and v1005_TO and v1006_TO;
	v1204_1_RS, v1204_1_DP, v1204_1_DG , v1204_1_DW , v1204_1_DR =
       Route(v1204_1_RC, (false -> pre v1204_1_S), 
      (false -> pre v1101_NL and pre v1102_NL and pre v1103_NL and pre v1104_NL),
       v1204_LS, v1204_2_WTC , v1204_2_GTC);


-- Route 1204(2) (Ends 8257)
-- Locks Points Normal
-- 1103/4
-- Locks Points Reverse
-- 1101/2
-- Tracks Clear Green : 1004 , 1009 , 1010 ,1011
-- Tracks Clear White

	v1204_2_WTC = false;
	v1204_2_GTC = v1004_TO and v1009_TO and v1010_TO and v1011_TO;
	v1204_2_RS , v1204_2_DP, v1204_2_DG , v1204_2_DW , v1204_2_DR = 
      Route( v1204_2_RC, (false -> pre v1204_2_S) ,
     (false -> pre v1101_RL and pre v1102_RL and pre v1103_NL and pre v1104_NL),
      v1204_LS, v1204_2_WTC, v1204_2_GTC);

-- Route 1206 (Ends A1203)
-- Locks Points Normal
-- 1103/4
-- 1101/2
-- Tracks Clear Green : 1010 , 1009 , 1008 , 1007
-- Tracks Clear White :

	v1206_WTC = false;
	v1206_GTC = v1010_TO and v1009_TO and v1008_TO and v1007_TO;
	v1206_RS , v1206_DP, v1206_DG, v1206_DW, v1206_DR	=
             Route(v1204_2_RC, (false -> pre v1204_2_S) ,
    (false -> pre v1101_NL and pre v1102_NL and pre v1103_NL and pre v1104_NL),
 v1206_LS, v1206_WTC, v1206_GTC);


-- Route 1205 (Ends A1203)
-- Locks Points Normal
-- 1101/2
-- Locks Points Reverse
-- 1103/4
-- Tracks Clear Green : 1005 , 1010 , 1009 , 1008 , 1007
-- Tracks Clear White

	v1205_WTC = false;
	v1205_GTC = v1005_TO and v1010_TO and v1009_TO and v1008_TO and v1007_TO;
	v1205_RS , v1205_DP, v1205_DG , v1205_DW , v1205_DR =
  	Route(v1205_RC, (false -> pre v1205_S) , 
      (false -> pre v1101_NL and pre v1102_NL and pre v1103_RL and pre v1104_RL),
      v1205_LS, v1205_WTC, v1205_GTC);

 
-- Routes Without Points
-- Route 1203
-- Tracks Clear Green : 1007, 8002
-- Tracks Clear White : 8004

	v1203_WTC = false;
	v1203_GTC = v1007_TO; 
	v1203_RS , v1203_DP, v1203_DG, v1203_DW, v1203_DR	
       = Route(v1203_RC, 
      (false -> pre v1203_S), true, v1203_LS, v1203_WTC , v1203_GTC );
	
-- Route 1201
-- Tracks Clear Green : 1002
-- Tracks Clear White : 1003
	
	v1201_GTC = v1002_CR;
	v1201_WTC = v1003_CR;
	v1201_RS , v1201_DP, v1201_DG, v1201_DW, v1201_DR	=
       Route(v1201_RC,(false -> pre v1201_S), true, v1201_LS, v1201_WTC , v1201_GTC);

-- Signals
-- A1201  (3 Aspect)

	v1201_LS , v1201_RED, v1201_WHI, v1201_GRE =
       Light3Aspect(v1201_DR, v1201_DW, v1201_DG);
	
-- A1202 (3 Aspect)
-- There is no control table entry for this signal
-- it may not actually be part of the scheme.
--	v1202_LS , v1202_RED, v1202_WHI, v1202_GRE =
--       Light3Aspect(v1202_DR , v1202_DW, v1202_DG);
	
-- A1203 (3 Aspect)
	
	v1203_LS , v1203_RED, v1203_WHI, v1203_GRE =
       Light3Aspect(v1203_DR , v1203_DW, v1203_DG);
	
-- 1204 (2 Aspect with Route Indicator and RS)
	
	v1204_LS , v1204_RED, v1204_GRE = Light2Aspect(v1204_2_DR or
       v1204_1_DR , ((v1204_1_DG and v1204_1_DW) or (v1204_2_DG and v1204_2_DW) ));
	
-- 1205 (2 Aspect with RS)

	v1205_LS , v1205_RED , v1205_GRE =
      Light2Aspect(v1205_DR, (v1205_DG and v1205_DW ));

-- 1206 (2 Aspect with RS)

	v1206_LS , v1206_RED , v1206_GRE =
      Light2Aspect(v1206_DR, (v1206_DG and v1206_DW ));

-- Fixed Reds  1207 , 1208 , 1209 , 1210

	v1207_D = FixedRed();
	v1208_D = FixedRed();
	v1209_D =	FixedRed();
	v1210_D =	FixedRed();

	v1204_1_A, v1204_2_A, v1206_A, v1205_A, v1203_A, v1201_A, 
      v1204_1_RC, v1204_2_RC, v1206_RC, v1205_RC, v1203_RC, 
      v1201_RC, v1204_1_S, v1204_2_S, v1206_S, v1205_S, v1203_S, v1201_S
	= RouteController(v1204_1_R, v1204_2_R, v1206_R , v1205_R, v1203_R,
         v1201_R , v1204_1_RS,  v1204_2_RS, v1206_RS , v1205_RS, v1203_RS , v1201_RS);

	
	-- Safety Conditions
	-- Points Locked when Route Set
	sPLRS_1206 = (not v1206_RS or (v1101_NL and v1102_NL and v1103_NL and v1104_NL));
	sPLRS_1204_1 = (not v1204_1_RS or (v1101_NL and v1102_NL and v1103_NL and v1104_NL));
	sPLRS_1204_2 = (not v1204_2_RS or (v1101_RL and v1102_RL and v1103_NL and v1104_NL));
	sPLRS_1205 = ( not v1205_RS or (v1101_NL and v1102_NL and v1103_RL and v1104_RL));
	
	-- No proceed aspect if route not set
	sNPA_1206 = (not v1206_GRE or v1206_RS);
	sNPA_1204 = (not v1204_GRE or #(v1204_1_RS , v1204_2_RS));
    sNPA_1205 = (not v1205_GRE or v1205_RS);
	
	
	-- No proceed aspect if train in route
	
	sNPATR_1206 = (not v1206_GRE or (v1010_TO and v1009_TO and v1008_TO and v1007_TO));
	sNPATR_1204_1 = (not (v1204_GRE and v1204_1_RS) or
                       (v1004_TO and v1005_TO and v1006_TO));
	sNPATR_1204_2 = (not (v1204_GRE and v1204_2_RS) or 
                      (v1004_TO and v1009_TO and v1010_TO and v1011_TO));
	sNPATR_1205 = (not v1205_GRE or 
                    (v1005_TO and v1010_TO and v1009_TO and v1008_TO and v1007_TO));

	-- Crashes can not occur
	
	nocrash = not v1004_CR;
	
	-- Trains coming into a piece of track should occupy it
	
	sTI1001   = true -> (not pre TrainIn) or v1001_TO;
	
	
tel

\end{verbatim}


\section{Modular Verification}

\subsubsection{Topological Verifcation}


\begin{verbatim}
node ModularVerification1(TrainIn : bool)
returns (sTrainIn, sTrainOcc, vTRACK1_TO , vTRACK1_O, vTRACK1_CR ,
vTRACK2_TO , vTRACK2_O, vTRACK_CR  : bool)

let		

		vTRACK1_TO , vTRACK1_O, vTRACK1_CR = Track_Segment1( TrainIn, false, false, true );  
		vTRACK2_TO , vTRACK2_O, vTRACK_CR = Track_Segment1(vTRACK1_O , false, false , true);


		-- Below are the safety conditions to verify that 
		-- two segments of track behave properly when connected together

		-- Trains Coming into a piece of track should occupy it
		
		sTrainIn   = true -> (not pre TrainIn) or vTRACK1_TO;

		-- If a train leaves a piece of track  it should cease
             -- to occupy this piece of track and occupy the next
		
		sTrainOcc = true -> (not pre vTRACK1_O) or 
            (vTRACK2_TO and (not pre TrainIn or vTRACK1_TO));
			
		
tel

\end{verbatim}


\begin{verbatim}

node ModularVerification2(TrainIn, Testpoint: bool)
returns (vTRACK_TO, vTRACK_O, vTRACK_O_2 , vTRACK_O_3, vTRACK_CR, sTP : bool)


let

	vTRACK_TO, vTRACK_O, vTRACK_O_2 , vTRACK_O_3, vTRACK_CR =
       Track_Segment3(TrainIn , false,
       false ,  false , true, false , Testpoint , not Testpoint);

		
		-- Points must influence which direction the train goes on the track.
		
		sTP =  not vTRACK_O_2 or  not Testpoint and not vTRACK_O_3 or Testpoint;

tel

\end{verbatim}

\begin{verbatim}

node ModularVerification3(TrainIn, RED , GREEN, WHITE : bool)
returns ( vTRACK1_TO , vTRACK1_O, vTRACK1_CR ,sRed : bool)

let

		vTRACK1_TO , vTRACK1_O, vTRACK1_CR = Track_Segment1( TrainIn, RED, GREEN, WHITE );  

			-- Trains dont leave the track if a Red light is showing
			
			sRed = not RED or vTRACK1_TO;

tel

\end{verbatim}

\subsection{Route Verification}




-- Route Verification

\begin{verbatim}
node ModularVerification4(RouteSet, PointsLocked, LightsSet,
                         W_track_clear, G_track_clear : bool)
returns(RouteSelected , DrivePoints, DriveGreen ,
         DriveWhite , DriveRed, SafetyRed : bool)
let

	-- Route(RouteCall, RouteSet, PointsLocked, LightsSet, W_track_clear, G_track_clear)
	
 RouteSelected , DrivePoints, DriveGreen , DriveWhite , DriveRed = 
      Route(false, RouteSet, PointsLocked, LightsSet, W_track_clear, G_track_clear);

	-- If the route is never called The red light and only the red light should be driven.

	SafetyRed = DriveRed and not DriveGreen and not DriveWhite;
	
	
tel
\end{verbatim}

\begin{verbatim}

node ModularVerification5(RouteCall, RouteSet , PointsLocked , LightsSet,
                           W_track_clear , G_track_clear : bool)
returns( RouteSelected , DrivePoints, DriveGreen ,
         DriveWhite , DriveRed, SafetyRed : bool)
let




	-- If the Route is called by but the track is not clear
     -- then the red aspect should be driven. 
	
      -- There are two possibilities for the formalisation of this safety condition.
      -- RouteSelected , DrivePoints, DriveGreen , DriveWhite , DriveRed = 
      --   Route(RouteCall, RouteSet , PointsLocked , LightsSet , false , false );
      -- SafetyRed = not RouteCall or (DriveRed and not DriveGreen and not DriveWhite);
	
	RouteSelected , DrivePoints, DriveGreen , DriveWhite , DriveRed = 
        Route(RouteCall, RouteSet , PointsLocked ,
              LightsSet , W_track_clear , G_track_clear );
	
      SafetyRed =  not( RouteCall and not W_track_clear and not G_track_clear)or 
                   (DriveRed and not DriveGreen and not DriveWhite);
	
tel
\end{verbatim}
\label{lst:Modver5}


\begin{verbatim}

node ModularVerification6(RouteCall, RouteSet , PointsLocked , LightsSet : bool)
returns(RouteSelected , DrivePoints, DriveGreen , DriveWhite ,
                                  DriveRed, SafetyCond : bool)
	
let


	RouteSelected , DrivePoints, DriveGreen , DriveWhite , DriveRed =
       Route(RouteCall, true , PointsLocked , LightsSet , true , true );
      
	-- If the track is clear and the Route is set then the Green light should show.


	SafetyCond =  true -> (not pre RouteCall) or  
               DriveGreen and not DriveWhite and not DriveRed;

tel

\end{verbatim}


\begin{verbatim}

node ModularVerification7(RouteCall, RouteSet , PointsLocked , LightsSet : bool)
returns(RouteSelected , DrivePoints, DriveGreen , DriveWhite ,
                                  DriveRed, SafetyCond : bool)
var

	foo : bool;
let

	foo = false -> if (pre foo or pre RouteCall) then true else false;
	

	RouteSelected , DrivePoints, DriveGreen , DriveWhite , DriveRed =
       Route(RouteCall, foo , PointsLocked , LightsSet , true , true );

	SafetyCond =  true -> (not pre RouteCall) or
       DriveGreen and not DriveWhite and not DriveRed;

tel
\end{verbatim}



\end{document}

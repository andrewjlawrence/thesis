\chapter{Background: Model Checking}
%%%There are many systems in the modern world that are built from communicating concurrent subsystems 
Model checking is a formal method that is typically used to reason about the correctness of finite systems with a discrete state space. Due to the nature of modern computers, such systems occur often in the area of computer science such as electronics or processes on a computer. It attempts to perform an exhaustive search of the state space to check that each state has a particular property. In this thesis is concerned with two types of model checking. One encodes the object being modelled as some form of state transition system before checking properties over that system specified in some temporal logic. This is called temporal logic model checking and was invented in the 1980s by two seperate research teams  Clarke and Emerson \cite{EM82} and Quielle and Sifakis \cite{JQ82}. The other encodes the system and the model checking problem over it as a SAT problem which is then decided using a SAT solver \cite{MS00}. The very nature of this formal method in that it is essentially an exhaustive search is also its downfall. A small increase in the complexity of the computer system can lead to a large increase in the state space. This exponential increase in the complexity of model checking a particular system is called the state \emph{space explosion problem}. Most improvements to the standard model checking algorithms are designed to reduce the impact of this problem. These optimisations typically exploit the structure of the system or properties such as symmetry to cut down the state space. Examples of modern model checkers include SPIN \cite{GH04} which performs on-the-fly model checking  based around the partial order reduction technique \cite{DP94}  and the probablistic model checker Prism \cite{MK11}. The Maude LTL Model Checker \cite{} also performs on-the-fly model checking and the underlying principles behind this will be discusses later in this chapter.

\section{Model Checkng Preliminaries}
In the following we shall describe some logical fundementals needed to understand model checking. Firstly we shall introduce a form of state transition system namely \emph{Kripke Structures} which are used capture the behaviour of finite discrete systems. Following this we shall introduce a language \emph{Linear Temporal Logic} which will enable us to formulate properties over these systems.
\medskip
\begin{mydef}[Kripke Structure]
Given a set of atomic propositions $AP$. We define a Kripke Structure $M$ to be a four tuple $(S, S_0, R, L)$
where:
\begin{itemize} 
\item $S$ is a finite set of states
\item $S_0$ is a set of initial states such that $S_0 \subseteq S$
\item $R$ is a total transition relation such that $R \subseteq S \times S$
\item $L: S \to 2^{AP}$ is a function that labels each state with the set of atomic propositions holding in that state.
\end{itemize}
\end{mydef}
\medskip
When we are formalising properties over a transition system such as a Kripke Structure it is useful from the point of view of safety to check which states a system can reach from a specific set of initial states. In practice it is rarely possible for a system to be safe for all states and therefore only states that are reachable from a set of safe initial states are considered. We define the set of reachable states inductively, all states in the initial set of states are reachable and a state is reachable if there is a transition to that state from a previous state that is also reachable.
\medskip
\begin{mydef}[Reachability]
Given a transition system $M = (S,S_0, R)$, we define $Reach_R(S, S_0)$ the set of states reachable from $S_0$ in $S$ via the transition relation $R$. Formally it is the least set of states such that the following holds: 

$$ Reach_R(S,S_0) := \{s \in S | s \in S_0 \vee R(r,s) \wedge r \in Reach_R(S,S_0)\} $$

\end{mydef}
\medskip
We define a model to be safe with respect to a safety property if for every state in its reachable set of states that safety property holds.
\medskip
\begin{mydef}[Model Safety]
We define the model represented by a transition system $M = (S,S_0, R)$ to be safe with respect to a safety property $\phi$ if the following holds:
$$\forall s \in Reach_R(S,S_0). s \models \phi $$

\end{mydef}
\medskip
\newcommand{\sseq}[2]{s_{{[}#1 \ldots #2 {]}} }

\begin{myremark}
Given a state $s$ and a set of states $S$ we will abbreivate set membership as a predicate:
 $$s \in S \leftrightarrow S(s)$$

We also  abbreivate  a sequence of states $s_n \ldots s_m$ as $\sseq{n}{m}$

\end{myremark}
\medskip
Knowing whether or not an unsafe state is reachable or not so useful from an engineering point of view as it does not provide any information as to how the unsafe state was reached. It is therefore desireable to be able to formulate a \emph{path} or sequence of states through a transition system. In the case that such an unsafe state is found, model checking techniques that check paths through the transition system will be able to provide a such a path as a \emph{counter example trace}.
\medskip
\begin{mydef}[Path]
Given a sequence of states $\sseq{0}{n}$ and a transition relation $R$ we define a path as follows:
 
$$path_R(\sseq{0}{n})  := \forall i \in \{0, \ldots, n-1 \} : R(s_i, s_{i+1})$$

\end{mydef}
\medskip
\subsection*{Linear Temporal Logic}

In order to perform model checking over a system we typically need a formal language that allows one to speak about time. We need to be able to formalise sentences such as "the next moment of time" and "all moments in time in the future". One such logic that allows us to formalise these statements is linear temporal logic (LTL)\cite{AP77}.  LTL belongs to a class of logics called temporal logics and is a sub-logic of Computional Tree Logic (CTL*) \cite{EE86}. The formulas of CTL* are built from path quantifiers and temporal operations. There are two path quantifiers, the for all computational paths quantifier \textbf{A} and the exists a computational path quantifier \textbf{E}. 

LTL operations can be used to speak about paths through a system specified as a Kripke structure.
a \emph{path} $\pi$ is a sequence of states $\sseq{0}{n}$ and a \emph{path formula} is one that holds in each given state of a path. 

Using these atomic propositions combined with the temporal operators it is possible to define a syntax for temporal logic formulae.
\medskip
\begin{mydef}[Syntax of Linear Temporal Logic]
Given a set of atomic propositions $AP$ we define an LTL path formula (pf) as follows:

\begin{itemize}
\item if $p \in AP$ then $p$ is a pf.

\item if $f$ and $g$ are pfs then $\star f$ and $f \circ g$ are pfs where $\star \in \{\neg,\mathbf{X},\mathbf{G}, \mathbf{F}\}$ and $\circ \in \{ \wedge,\vee,\textbf{R},\textbf{U} \}$.
\end{itemize}

\end{mydef}
\medskip
We shall now look at the semantics of LTL firstly using an informal description of the temporal operations from which LTL is constructed and secondly by giving a formal semantics for LTL. The following is a description of the 5 LTL operations over paths of a Kripke Structure.
\begin{itemize}
\item \textbf{X} $f$ : The property $f$ holds in the \emph{next} moment of time.
\item \textbf{G} $f$ : The property $f$ is \emph{globally} true. i.e. it holds for all times on all paths. 
\item \textbf{F} $f$ : The property $f$ is \emph{finally} true. i.e. there exists a time such that the property $f$ holds on a path.
\item $f$ \textbf{U} $g$ : For all paths the property globally $f$ holds \emph{until} property $g$ holds. 
\item $f$ \textbf{R} $g$ : f holds up to and including the point when $g$ holds.
\end{itemize}

%%%
%%% We need to formalise the following definition. Semantics of Linear Temporal Logic Formula?
%%% 

The semantics of LTL is defined inductively for all operations in the following, it is however possible to define the semantics just in terms of the $\mathbf{X}$ and $\mathbf{U}$ operations and to obtain the semantics for the other operations via equivalences.

 \medskip
\begin{mydef}[Semantics of Linear Temporal Logic]
We define the semantics of a linear temporal logic formula $\phi \in LTL(AP)$ in terms of a satisfaction relation $$M,s \models \phi$$ over a Kripke structure $M = (S, S_0,R,L)$ and a state $s \in S$ as follows:

$$K,s \models \phi$$ holds if and only if $\forall \pi \in Path(K, s)$, $K,\pi \models \phi$ holds.

We define what it means for $K, \pi \models \phi$ to hold inductively as follows:

\begin{comment}

\begin{itemize} 
\item $K,\pi \models \phi$ always holds.

\item For all propositional formulae $p \in AP$ the following always holds:
     $$K,\pi \models p \leftrightarrow p \in L(\pi(0))$$

\item For all LTL formulas $\mathbf{X} \phi \in LTL(K)$ the following always holds:
$$ K, \pi \models \mathbf{X} \phi \leftrightarrow K, succ;\pi \models \phi $$

where $succ$ is a successor function $succ: N \_ \to N$ such that $succ;\pi(n) = \pi(succ(n)) = \pi(n + 1)$.

\item For all LTL formulas $\phi \mathbf{U} \psi \in LTL(K)$ the following always holds:
$$K, \pi \models_{LTL} \phi \mathbf{U} \psi \leftrightarrow$$
$$\exists n \in N.$$
$$(K,succ^n; \pi \models_{LTL} \psi) \wedge (\forall m \in N. m < n \to K, succ^m;\i \models_{LTL} \phi)$$

\item For all LTL formulas $\neg \phi \in LTL(AP)$ the following always holds:
$$K,\pi \models_{LTL} \neg \phi \leftrightarrow K,\pi \not\models \phi $$

\item For all LTL formulas $\phi \vee \psi \in LTL(AP)$ the following always holds
$$K,\pi \models_{LTL} \phi \vee \psi \leftrightarrow K,\pi \models_{LTL} \phi \ \mathrm{or} \ K,\pi \models_{LTL} \psi $$

\end{itemize}

\end{comment}

\begin{center}
\begin{tabular}{l | l  c  l}
1. & $M, \pi \models p$  & $\Leftrightarrow$ & $p \in L(\pi(0))$ \\

2. & $M, \pi \models \neg \phi $ & $\Leftrightarrow$ & $M,\pi \not\models \phi$ \\
3. & $M, \pi \models \phi \vee \psi$ & $\Leftrightarrow$ & $M,\pi \models \phi$ or $M,s \models \psi$ \\
4. & $M, \pi \models \phi \wedge \psi$ & $\Leftrightarrow$ & $M,\pi \models \phi$ and $M,s \models \psi$ \\ 
5. & $M,\pi \models \phi$ & $\Leftrightarrow$ & $M,\pi(0) \models \phi$ \\
6. & $M,\pi \models \mathbf{X} \phi$ & $\Leftrightarrow$ & $M, succ;\pi \models \phi$ \\
7. & $M, \pi \models \phi \mathbf{U} \psi$ & $\Leftrightarrow$ & $\exists n \in \mathbb{N}. M,succ^n; \pi \models \psi \wedge \forall m \in \mathbb{N}. m < n \to succ^m; \pi \models \phi$ \\
8. & $M,\pi \models \mathbf{F} \phi$ & $\Leftrightarrow$ & $\exists n \in \mathbb{N}. M,succ^n;\pi \models \phi$ \\
9. & $M,\pi \models \mathbf{G} \phi$ & $\Leftrightarrow$ & $\forall n \in \mathbb{N}. M, succ^n;\pi \models \phi$ \\
10. & $M,\pi \models \phi \mathbf{R} \psi$ &$\Leftrightarrow$&  $\forall n,m \in \mathbb{N}. i < j \to M,succ^i; \pi \not\models \phi \to  M,\succ^j \models \psi$ 
\end{tabular}
\end{center}

where $succ$ is a successor function $succ: \mathbb{N} \_ \to \mathbb{N}$ such that $succ;\pi(n) = \pi(succ(n)) = \pi(n + 1)$.

\end{mydef}
There are equivalences between the following formulas containing $\textbf{G}$ and $\textbf{R}$ and formulas containing $\mathbf{F}$ and $\mathbf{U}$.

$$\textbf{G} \phi \equiv False \, \mathbf{R} \, \phi $$
$$\textbf{F} \phi \equiv   True \, \mathbf{U} \, \phi$$

Combined  with the equivalences for obtaining negative normal form.

\begin{mydef}[LTL Negative Normal Form]
The negative normal form of an LTL formula can be obtained by applying the following equivalences:
$$\neg True \equiv False$$
$$\neg False \equiv True$$
$$\neg \neg \phi \equiv \phi$$
$$\neg (\phi \vee \psi) \equiv \neg \phi \wedge \neg \psi $$
$$\neg (\phi \wedge \psi) \equiv \neg \phi \vee \neg \psi$$
$$\neg \mathbf{X} \phi \equiv \mathbf{X} \neg \phi$$
$$\neg (\phi \mathbf{U} \psi) \equiv (\neg \phi) \mathbf{R} (\neg \psi)$$
$$\neg (\phi \mathbf{R} \psi \equiv (\neg \phi) \mathbf{U} (\neg \psi)$$
from left to right until no further equivalence is applicable.
\end{mydef}

\section{Model Checking Propositional Safety Properties Using a SAT Solver}


In the following we shall introduce two general types of model checking  which are applied to check propositional safety properties over discrete systems using a SAT solver \cite{MS00,EC01}. The first is inductive verification, which is split into two steps consisting of a base case and a step case.   The second that we shall discuss is bounded model checking which attempts to verify a property of the system with an exhaustive search up to a particular depth. There are other types of model checking including temporal induction which has the benefits of both inductive verification and bounded model checking however suffers in terms of speed and the size of the resulting problem. All of these problems are formulated in propositional logic and then translated into CNF using the Tseitin Translation.



\subsection*{Inductive Verification}

Inductive verification attempts to use induction over the transition relation in order to perform verification formulated in such a way that a SAT solver can be used to solve the problem \cite{MS00}. The base case checks that the safety property holds in all intial states of the system. The step case checks that if the system is in a state in which the safety property holds and a transition occurs then the resulting state is also safe, for all possible safe states and transitions out of those states. This method has the advantage over other model checking techniques that it is fast and can verify then entire state space. The disadvantage is that if the safety property is violated then you obtain a pair of states as a counter example instead of a sequence of states that form a counter example trace.
\medskip
\begin{mydef}[Inductive Verification]
Given a transition system $M$ and a safety property $\phi$ we define inductive verification as follows:

$$indbase_M(\phi)  := \forall s_0. S_0(s_0)  \to s_0 \models \phi $$
$$indstep_M(\phi) : \forall s,s' \in S \wedge R(s, s') \to s \models \phi  \to s' \models \phi$$

\end{mydef}
\medskip
\subsection*{Bounded Model Checking}
One of the most important model checking techniques in terms of industrial applications is bounded model checking. All possible paths through the system are safety property checked up to a certain bound. This approach is most useful for falisification checking i.e. finding errors and their causes. It is scaleable to industrial applications as they typically exhibit all of their possible behaviours after a short number of time steps. We present a formalisation that is geared towards the application of SAT solvers \cite{EC01}. In the following we make use of the abbreviation $\sseq{n}{m} \models \phi \leftrightarrow s_n \models \phi \wedge \ldots \wedge  s_m \models \phi$ when writing that a formula holds $\phi$ in a sequence of states $\sseq{n}{m}$.
\medskip
\begin{mydef}[Bounded Model Checking]
Given a transition system $M$, a safety property $\phi$ and a bound $k$ we define bounded model checking $bmc_M(\phi,k)$ as follows:

$$bmc_M(\phi,k):= \forall \sseq{0}{k} \subseteq S \to  path_R(\sseq{0}{k}) \to S_0(s_0) \to \sseq{0}{k} \models \phi$$
\end{mydef}
\medskip
\textbf{Note to self: I need to add something on loop freedom and inclusion checking for sat based model checking, I also need to describe the LTL model checker in Maude which uses Buchi automaton.}  

\section{On-the-fly Model Checking of Linear Temporal Logic Properties}
The Maude LTL model checker performs on-the-fly model checking \cite{CC91} of an LTL property $\phi$.
This process of mode checking linear temporal logic involves first  negating the linear temporal  logic property, placing it in negation normal form and then constructing a B{\"u}chi automata for that property that accepts the language of counter examples $\neg \phi$. The product of the Buchi automata and the state transition system is constructed and a double depth first search is applied to search for strongly connected components with a state in which $\neg \phi$ holds.

\subsection{B{\"u}chi Automata}
\newcommand{\Buchi}{B{\"u}chi\xspace}
Most applications of model checking typically search for a finite counter example trace in the set of infinite traces of the system. The class of automatons which can accept infinite words is called the $\omega$-automata and the simplest class $\omega-automata$ is \Buchi automata. These \Buchi are a minor modification of finite state automaton in that they have a finite alphabet and a finite set of states but instead of a set of final states the automata have a set of \emph{accepting states}.

\begin{mydef}[B{\"u}chi Automata]
We define a \Buchi automaton $A$ is a five-tuple $(L, S,T,S_0, F)$ where:
\item $L$ is a finite alphabet
\item $S$ is a finite set of states
\item $T \subseteq S \times L \times S$ a transition relation
\item $S_0 \subseteq S$ a set of initial states 
\item $F \subseteq S$ is an acceptance condition. The automaton A accepts only the infinite runs which contain at least one state occuring in $F$
\end{mydef}

The translation of the LTL property into a \Buchi automata takes 3 steps following the approach of Gastin and Oddoux \cite{}:

\begin{enumerate}
\item Construct a very weak alternating (VWA) automata
\item Convert this VWA automata into a generalized \Buchi automata
\item Convert the generalized \Buchi automata into a standard \Buchi automata
\end{enumerate}

At every step the automaton is simplified by removing unreachable states, subsumed transitions and combining equivalent states. In order to perform model checking we need to be able to find accepting runs of the automata since these accepting runs are infinite they must contain a loop through the automata. Therefore the model checking algorithm partitions the \Buchi automaton into its strongly connected components. A set of states in a \Buchi automaton is a \emph{strongly connected component} (SCC) if and only if every state in that set is reachable from every other state in that set. 

The model checking algorithm performs a search for strongly connected components and tests to see whether the the acceptance condition is empty.

Fair strongly connected component?



\textbf{Note to self: Do I need to describe the differences between SAT based model checker and other approaches?}

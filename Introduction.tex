 \thesischapter{Introduction}


\section{Introduction}

This thesis is concerned with the application of formal methods within the Railway Domain. Firstly we present a new approach to develop verified SAT solving algorithms and which have been applied the verification of a real world train control system: the solid state interlocking. Secondly we present an approach to formalise the European Rail Traffic Management System (ERTMS) and apply a model checker to verify the systems safety.


\section{Aim}
We have 2 sets of aims. The first set of aims is in regard to the development of verified SAT solving algorithms and their application to the verification of solid state railway interlocking programs: 

\begin{itemize}

\item Formalize the DPLL proof system and a proof of its completeness.

\item Extract a standard DPLL SAT algorithm from the formalisation into a functional programming language and test its performance

\item Modify the formalisation and completeness proof  of the DPLL proof system so that it captures the behaviour conflict driven clause learning SAT algorithms.

\item Extract a clause learning SAT algorithm from the formalisation  of the modified DPLL proof system and show that clause learning increases the efficiency of the solver.

\item Apply verified SAT algorithms to the verification of solid state railway interlocking programs. 

\end{itemize}

Our second set of aims is in regard to the formal specification and verification of the European Rail Traffic Management System (ERTMS) using Real Time Maude.

\begin{itemize}

\item Formalise ERTMS as a hybrid automata.

\item Formalise ERTMS as a Real Time Maude specification.

\item Verify ERTMS using Real Time Maude's linear temporal logic model checker.

\end{itemize}




\section{Thesis Outline}



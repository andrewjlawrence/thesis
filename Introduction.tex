 \thesischapter{Introduction}


\section{Introduction}

This thesis is concerned with the formalisation and verification of SAT algorithms and their application to model checking in the railway domain. 


\section{Aim}
We intend to develop and apply verified SAT algorithms to the verification of train control systems. Specifically we aim to: 

\begin{itemize}

\item Formalize the DPLL proof system and a proof of its completeness.

\item Extract a standard DPLL SAT algorithm from the formalisation into a functional programming language and test its performance

\item Modify the formalisation and completeness proof  of the DPLL proof system so that it captures the behaviour conflict driven clause learning SAT algorithms.

\item Extract a clause learning SAT algorithm from the formalisation  of the modified DPLL proof system and show that clause learning increases the efficiency of the solver.

\item Apply verified SAT algorithms to the verification of traditional railway interlockings. 

\item Formalise of a Hybrid train control system that contains both discrete and continuous data.

\item Apply verified SAT algorithms to the verification of a modern train control system.

\end{itemize}





In this thesis we will be exploring the verification of railway 
interlockings with a
view to practical application in industry and investigating the use of a
commercial piece of software \scade \ suite from Esterel Technologies. We intend to answer the question: How well does \scade \ suite perform the
verification of code and what ways is it applicable to the verification of
railway interlockings? We explore the application of model checking where
we specifically concentrate on two different modelling approaches.

In the first approach we translate
existing specifications of railway interlockings written in so-called 
Ladder Logic
into \scade \ and verify them. To do this a new formalism is introduced 
to capture
the semantics of ladder logic using labelled transition systems.
We first use \scade \ to verify a small toy ladder logic program. In our 
first
attempt the translation was done manually. To scale up this approach a tool was created to automatically translate
ladder logic into \scade \ language. Our work on formalizing ladder logic builds on two
previous MRes projects by Kanso \cite{KKanso} and James \cite{PJames}. The
problem of verifying ladder logic was first approached by Kanso \cite{KKanso}
who was the first to systematically describe ladder logic and who developed
a prototype translation and verification tool. James \cite{PJames} expanded the
functionality of the prototype tool by adding model checking techniques which could be
applied to the verification problem. We have applied \scade \ to the example case studies considered in these
two previously completed MRes Projects. These include two real world
railway control systems of considerable size and complexity.

The second approach presented in this project was driven by the desire 
of industry to change
the way they model their systems. There is a drive away from low level
languages towards more powerful high level languages. In a bid to drive down
costs the concept of reuse is important. Industry would like to have a 
toolkit
of components and pre-verified modules from which they can model new
railways. This led to the development of a new modelling approach which
embodies these points. An example railway was modelled to provide a case 
study
into its use. Verification was then performed on our railway example, on
individual components and on modules of components.

\section{Thesis Outline}

In Chapter 2 we will introduce a large amount of background information
regarding the railway, its history and composition. We will give an
introduction to ladder logic and describe the operation of the Westrace
Railway Interlocking.

In Chapter 3 we will discuss the theoretical background under-pinning the
\scade \ suite. We will explore St{\aa}marck's Saturation method giving both an
informal explanation using several examples and a formal explanation providing
the theoretical background and implementation details. This is followed by a survey of another technique used to decide the satisfiability of
propositional formulae namely binary decision diagrams. We provide a
formal introduction to binary decision diagrams as well as a previously known
result regarding their complexity.

In Chapter 4 we will explore the verification of ladder logic programs. We
begin by introducing a toy example ladder logic program for a pelican
crossing. Throughout this chapter we present several possible formalisations of
this in the \scade \ language. A new method is presented which captures the semantics of
ladder logic programs using labelled transition systems. We then present an approach for the
verification of ladder logic programs using the \scade \ suite. This includes a
automatic translation of ladder logic programs into the \scade \
language. We follow this with a discussion of invariants and their addition to
\scade \ models. Finally we perform a comparison between \scade, the tool
produced by James \cite{PJames} and the KIND safety property verifier.

In Chapter 5 we will introduce a new modelling approach to capture railway
domain in  a concrete and modular fashion. This begins with a discussion of
components modelled, some motivation for the modelling and a description of
their behaviour. Following this discussion an example is presented of an
abstract railway modelled using the aforementioned components.  We then have a 
discussion of the verification of our new modelling approach. This includes the
verification of topological properties and modules of components. We conclude
this chapter with a comparison between the first approach and this new
modelling approach. 

Finally in Chapter 6 we provide a conclusion to this thesis. We provide a
summary of the work performed and the results achieved. Then we conclude by
discussing possible directions for future work.

The Appendix is split into two sections. The first contains rules for several proof systems including
St{\aa}lmarck's saturation method. The second contains the \scade \ language
code for the concrete modelling approach.

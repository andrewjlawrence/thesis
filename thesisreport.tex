\documentclass{article} 
\usepackage{amsmath,amssymb, stmaryrd}
\usepackage{pdflscape}
\usepackage{graphicx}
\usepackage{wrapfig}
\usepackage{tikz}
\usepackage{bussproofs}
\usepackage{comment}
\usepackage{amsthm}
\usepackage{enumitem}
\usetikzlibrary{arrows,decorations.pathmorphing,backgrounds,fit}
\usetikzlibrary{shapes.geometric}
\usepgflibrary{shapes.geometric}
\usetikzlibrary{shapes.symbols}
\usepgflibrary{shapes.callouts}
\usetikzlibrary{shapes.callouts}

\newtheorem{definition}{Definition}
\newtheorem{lemma}{Lemma}
\newtheorem{theorem}{Theorem}
% \newenvironment{proof}[1][Proof]{\begin{trivlist}
% \item[\hskip \labelsep {\bfseries #1}]}{\end{trivlist}}



\title{Verification of Train Control Systems: Tools and Techniques (Completion Report)}
\date{\today}


%%% Authors are given together with their institut.

\author{ 
   Andrew Lawrence}

\begin{document}
\maketitle
This is an update on the  previous completion report submitted at the end of 2013.



\section{Progress on the Verification of ERTMS}
In the previous thesis report I stated that I would verify the European Rail Traffic Management System (ERTMS) using SAT based model checking. I have since discovered that my extracted SAT solvers are not capable of doing this as they do not implement the low level interface needed to connect with a model checker. The decision was therefore made to use the Real Time Maude tool to specify the ERTMS system and verify properties over it. I have specified the behaviour of the trains, interlocking and radio block processor and combined the 3 specifications together to form the Pentagon example described in our section on formalising ERTMS. I have made use of the simulation capabilities of the system and plotted graphs which validate its behaviour. Finally I verified safety properties such as "two trains do not crash" and "two trains do not have overlapping movement authorities" in our ERTMS example using the Maude LTL Model Checker.

\section{Thesis Progress}
The thesis is approximately two thirds completed. The following is the current structure of the thesis: 
\begin{enumerate}
\item  Introduction  
\begin{enumerate}  [label*=\arabic*.]
\item  Motivation 
\item  Summary of Results
\item  Thesis Outline
\end{enumerate}
\item Background: Logic and Program Extraction 
\begin{enumerate}[label*=\arabic*.]
\item Formal Foundations of Minlog 
\item  Other Theorem Provers with Program Extraction Capabilities
\end{enumerate}
\item Background: Resolution and the Satisfiability Problem 
\begin{enumerate}[label*=\arabic*.]
\item The Satisfiability Problem
\item The DPLL Algorithm
\item Conflict Driven Clause Learning Algorithms
\item Resolution
\item Translation to CNF
\item The Future: Satisfiability Modulo Theory
\end{enumerate}
\item Extracting Verified Decision Procedures 
\item Extracting a Clause Learning SAT Algorithm
\item Background: Traditional Railway Control Systems 
\begin{enumerate}[label*=\arabic*.]
\item A History of Railway Signalling and Control Systems 
\item Invensys Rail 
\item An Overview of the Railway Domain 
\item Ladder Logic 
\item Previous Work in this Field
\end{enumerate}
\item Background: Model Checking 
\begin{enumerate}[label*=\arabic*.]
\item Model Checkng Preliminaries 
\item Model Checking Propositional Safety Properties Using a SAT Solver
\item On-the-fly Model Checking of Linear Temporal Logic Properties 
\end{enumerate}
\item The Application and Comparison of X-SAT on Benchmark Problems 
\item Background: The European Rail Traffic Management System 
\begin{enumerate}[label*=\arabic*.]
\item The European Rail Traffic Management System 
\item Typical ERTMS Scenario
\item Previous Work: Attempts to Verify ERTMS
\end{enumerate}
\item Formalising the European Rail Traffic Management System Using Hybrid
Automata 
\begin{enumerate}[label*=\arabic*.]
\item Hybrid Automata 
\item Pentagon Example 
\item Semantics of Hybrid Automata
\item Automaton Validation 
\item Conclusion 
\end{enumerate}
\item The Modelling of the European Rail Traffic Management System Using
Real Time Maude 
\begin{enumerate}[label*=\arabic*.]
\item Maude 
\item Real Time Maude 
\item Modelling the European Rail Traffic Management System 
\end{enumerate}
\item Simulating and Verifying the European Rail Traffic Management System
Using Real Time Maude
\begin{enumerate}[label*=\arabic*.]
\item The Maude Linear Temporal Logic Model Checker 
\item Model Checking the European Rail Traffic Management System 
\item Conclusion
\end{enumerate}
\end{enumerate}

The results of my thesis can be seperated into two categories. The first contains those that deal with the verification of decision procedures and can be found in chapters 4, 5 and 8. The second category of results deals with the verification of ERTMS and is spread between chapters 10, 11 and 12. I have written most of the background chapters as well as sections on formalising and verifying ERTMS. The work on extracting verified both DPLL and Clause Learning SAT algorithms is written up seperately and needs to be updated and integrated into the thesis.

\section{Current Work}
I am currently reviewing and correcting the formalisation of ERTMS as a hybrid automaton along with the proofs that validate them. I am also in the process of making the background chapters consistent with each other so that they fit together nicely. We are planning on submitting an abstract to the Workshop on Algebraic Development Technqiues, in the middle of June, on the formalisation and verification of ERTMS. Finally I intend to integrate the work regarding the extraction SAT algorithms into the thesis along with an appendix containing their formalisation.
 
\end{document}

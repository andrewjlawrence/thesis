\mathchardef\mhyphen="2D
\newcommand{\mathass}[1]{\mathrm{[}#1 {]}}
\thesischapter{Background: Logic and Program Extraction}{Background: Logic and Program Extraction}
\chaptermark{Logic and Program Extraction}
%%%We use logic to formally reason about the world around us in an abstract way. At the most basic level %%%we are creating abstract propositions, sentences from these propositions and then using formal rules %%%to deal with these propositions and to derive new information or insight. In the following we give a %%%general introduction to logic and describe some of the formalisms used in this thesis. This starts with %%%syntax and semantics of propositional and first order logic.
This thesis is concerned with using formal methods to verify the correctness of programs. This chapter is concerned one technique in particular that is used to obtain verified programs, namely \emph{program extraction} from proofs as implemented by the interactive theorem prover Minlog. The theorem typically specifies the inputs of the system as assumptions and the output of the program as the statement to be proven. A constructive proof is given which describes how to transform the inputs into the required outputs. In the Minlog system the proof is stored as a proof term which can interpreted as a program in the lambda calculus. Once such a proof term has been obtained, it can be automatically translated by the Minlog system into a variety of different functional programming languages. This is due to the Curry-Howard correspondence \cite{HC34, HC58, WH80} which states that every constructive proof can also be seen as a program. It enables a proof system such as natural deduction to be directly interpreted as a model of computation i.e. the lambda calculus.  A general introduction to program extraction from proofs can be found in \cite{SW11}. This thesis assumes a general background in logic, a good general introduction to logic from a proof theoretic perspective is \cite{HandBookProof}. The following definitions for program extraction have been obtained from \cite{HS14}.

\subsection*{Related Work in the Area of Program Extraction}
There are a number of case studies in the area of extracting programs from proofs. The Minlog system has been used extraction of a normalisation-by-evaluation algorithm for the lambda calculus \cite{UB06}. There is also considerable work in the area of exact real number computation \cite{UB08} which has made use of Coalgebras and Coinduction \cite{UB11}. 


\subsection*{Other Theorem Provers with Program Extraction Capabilities}
While Minlog is specifically geared towards extracting programs from proofs there are other examples of theorem provers with program extraction capabilities. An early example of work on a theorem prover with program extraction capabilities is the Nuprl system \cite{RC86}. There are also other mature theorem provers with program extraction capabilities such as Coq \cite{YB04}, which is based on the Calculus of Inductive Constructions, and Isabelle  \cite{TN99,LP94}, which is a generic theorem prover with extensions for many logics. More recently a number of theorem provers based on dependent types \cite{PM80} have emerged including Agda \cite{AB09} and IDRIS \cite{EB11}.
\begin{comment}
\section{Propositional logic}
One of the most fundamental forms of logic is propositional logic which contains operations called connectives that allows one to connect propositional variables to form sentences.
\begin{mydef}[Propositional Formulae]
Given a set of atomic proposition names $AP$ we inductively define the set of propositional formulae $Prop$ as follows:
\begin{itemize}
\item If $a$ is an atomic proposition $a \in AP$ then $a \in Prop$.

\item if $p_1$ is a propositional formula $p_1 \in AP$ then $\neg p_1 \in AP$.

\item given two propositional formulae $p_1$ and $p_2$ then $p_1 \circ p_2 \in AP$ where $\circ$ is a propositional connective $\circ \in \{ \wedge,\vee,\to \} $.
\end{itemize} 
\end{mydef}
In order to define the semantics of propositional logic we need the notion of a truth assignment which is a function $\tau : \{True,False\}^n \to \{True, False\}$ that gives a logical assignment to the set propositional variables $V$ which is of size $n$. This function can then be used to describe another function $\bar{\tau} : P \to \{True, False\}$ that given a propositional formula $P \in AP$ calculates a truth assignment for that formula based on $\tau$.

We assigned these sentences of propositions a value of either True or False using a truth assignment 
\begin{center}
\begin{tabular}{c | c | c | c | c | c | c} \hline
$A$ & $B$ & $\neg A$  & $A \wedge B$ & $A \vee B$ & $A \to B$ & $A \leftrightarrow B$  \\ \hline
 T & F & F  & T & T & T & T \\ \hline 
 T & T & F  & F & T & F & F \\ \hline
 F & F & T  & F & T & T & F \\ \hline
 F & T & T  & F & F & T & T \\ 
\end{tabular}
\end{center}

\begin{mydef}[Satisfaction Relation for Propositional Logic]
Given a consistent valuation $\Gamma$ and two propositional formulae $\phi, \psi \in PROP$ we define a satisfaction relation $\models$ between a valuation and a formula as follows:
\begin{itemize}
\item $\Gamma \models \top$
\item $\Gamma \not\models \bot$
\item $\Gamma \models v$ iff $+v \in \Gamma$
\item $\Gamma \not\models v$ iff $-v \in \Gamma$
\item $\Gamma \models \neg \phi$ iff $\Gamma \not\models \phi$
\item $\Gamma \models \phi \wedge \psi$ iff $\Gamma \models \phi$ and $\Gamma \models \psi$
\item $\Gamma \models \phi \vee \psi$ iff $\Gamma \models \phi$ or $\Gamma \models \psi$
\end{itemize}

\end{mydef}

\section{First Order Logic}
\begin{mydef}[Syntax of First Order Logic]
Blah

\end{mydef}

\begin{mydef}[Semantics of First Order Logic]

\end{mydef}

\end{comment}
\section{Natural Deduction}
In the following sections we shall discuss in more detail the underlying logical foundations on which the interactive theorem prover Minlog is based. The proof calculus used by Minlog is the natural deduction proof system which was introduced by Gentzen in 1939 \cite{GG69}. He was looking to capture the method by which humans naturally reason in a logical formalism.  The resulting system is very general and allows for new facts to be deduced from what is previously known in an manner designed to be intuitive for mathematicians.  The particular version of natural deduction used is the variant for minimal logic which is suited for program extraction as its constituent proofs are constructive. Minlog's term calculus is based on the simply typed $\lambda$-calculus over free algebras (examples are natural numbers, lists, trees,...), and the logic is extended by strictly positive inductive definitions. Hence Minlog contains, for instance, Heyting Arithmetic in finite types, $\mathrm{HA}^\omega$, and an intuitionistic version of finitely iterated inductive definitions, $\mathrm{ID}^{<\omega}$. However note that type theoretic systems such as Agda or Coq are not captured by Minlog, as Minlog does not have dependent types. In the following, we will call Minlog's underlying formal system $\mathrm{HA}^\mu$ since it can be 
viewed as $\mathrm{HA}^{\omega}$ extended by inductive definitions. We shall now present the natural deduction proof system  for first order logic together with the proof terms derived from each rule:  \medskip
%
\begin{mydef}[Natural Deduction for Minimal Logic] 
The natural deduction proof system for minimal logic  consists of the following derivation terms for $\wedge$, $\to$ and $\forall$: 
\begin{center}

\begin{tabular}{| c | c |} \hline
derivation & term \\
\hline 
\raisebox{-1\height}{$u:A$} & \raisebox{-1\height}{$u^A$} \\ [3ex] \hline
\raisebox{-1.2\height}{
\AxiomC{$|M$}
\noLine
\UnaryInfC{$A$}
\AxiomC{$|N$}
\noLine
\UnaryInfC{$B $}
\RightLabel{$\wedge^+$}
\BinaryInfC{$A \wedge B$}
\DisplayProof} &  \raisebox{-2.8\height}{$\langle M^A, N^B \rangle^{A \wedge B}$} \\ [10ex]  \hline
\raisebox{-1.5\height}{
\AxiomC{$|M$}
\noLine
\UnaryInfC{$A \wedge B$}
\RightLabel{$\wedge^-_0$}
\UnaryInfC{$A$}
\DisplayProof \hspace{10pt}
\AxiomC{$|M$}
\noLine
\UnaryInfC{$A \wedge B$}
\RightLabel{$\wedge^-_1$}
\UnaryInfC{$B$}
\DisplayProof} & \raisebox{-2.8\height}{$(M^{A \wedge B} 0)^A \quad (M^{A \wedge B} 1)^B$} \\ [10ex] \hline

\end{tabular}
\end{center}

\begin{center}
\begin{tabular}{|c|c|} \hline

derivation & term \\ \hline

\raisebox{-1\height}{
\AxiomC{$\mathass{u: \,A}$}
\noLine
\UnaryInfC{$| \,M$}
\noLine
\UnaryInfC{$B$}
\RightLabel{$\to^+$}
\UnaryInfC{$A \to B$}
\DisplayProof} & \raisebox{-3.2\height}{$(\lambda u^AM^B)^{A \to B}$} \\ [12ex] \hline

\raisebox{-1\height}{
\AxiomC{$|M$}
\noLine
\UnaryInfC{$A \to B$}
\AxiomC{$|N$}
\noLine
\UnaryInfC{$A$}
\RightLabel{$\to^-$}
\BinaryInfC{$B$}
\DisplayProof} & \raisebox{-1.2\height}{$(M^{A \to B} N^{A})^B$} \\ [5ex] \hline



\raisebox{-1\height}{
\AxiomC{$|M$}
\noLine
\UnaryInfC{$A$}
\RightLabel{$\forall^+x \ (var.cond.)$}
\UnaryInfC{$\forall x A$}
\DisplayProof} & \raisebox{-2.8\height}{$(\lambda x M^A)^{\forall x A} \, (var.cond.)$}  \\ [10ex] \hline

\raisebox{-1\height}{
\AxiomC{$|M$}
\noLine
\UnaryInfC{$\forall x. A$}
\AxiomC{t}
\RightLabel{$\forall^-$}
\BinaryInfC{$A[x:=t]$}
\DisplayProof} & \raisebox{-2.8\height}{$(M^{\forall x A}t)^{A[x:=t]}$} \\ [10ex] \hline


\end{tabular}
\end{center}

The $\forall^{+}$ rule has the condition that $x$ is not free in $A$. 

\end{mydef}
\medskip
Minlog also allows for the use of the \emph{ex falso quodlibet} rule to derive falsity in a proof. Such proofs do not contain any computational content and it is typically used to prove properties about computations that are never executed by the program.\\
\medskip
\begin{mydef}[The Ex Falso Quodlibet Rule] \hspace*{0pt} \\
\begin{center}
\AxiomC{$\bot$}
\LeftLabel{efq}
\UnaryInfC{$A$}
\DisplayProof 
\end{center}
\end{mydef}

The stability rule is also included in the Minlog system, although I have not used it in any proofs in this thesis, it is used to obtain intuitionistic logic. The Minlog system contains mechanisms to perform program extraction from classical proofs \cite{UB95}. 
\medskip
\begin{mydef}[The Stability Rule] \hspace*{0pt} \\
\begin{center}
\AxiomC{$A \to \bot \to \bot$}
\LeftLabel{stab}
\UnaryInfC{$A$}
\DisplayProof 
\end{center}
\end{mydef}


The natural deduction proof system for minimal logic previously introduced is extended to classical logic by the addition of the following rules for $\vee$ and $\exists$. Following the description of these rules we describe their Minlog specific implementation details.


\begin{center}
\begin{tabular}{| c | c |} \hline

derivation & term \\ \hline

\raisebox{-1\height}{
\AxiomC{$t$}
\AxiomC{$|M$}
\noLine
\UnaryInfC{$A[x:=t]$}
\RightLabel{$\exists^+$}
\BinaryInfC{$\exists x A$}
\DisplayProof} & \raisebox{-2.2\height}{$(\exists^+_{x,A} t M^{A[x:=t]})^{\exists x A}$} \\ [10ex] \hline

\raisebox{-1\height}{
\AxiomC{$|M$}
\noLine
\UnaryInfC{$\exists x A$}
\AxiomC{$\mathass{u:A}$}
\noLine
\UnaryInfC{$|N$}
\noLine
\UnaryInfC{$B$}
\RightLabel{$\exists^- (var.cond.)$}
\BinaryInfC{$B$}
\DisplayProof} & \raisebox{-3.8\height}{($M^{\exists x A}(u^A.N^B))^B$ (var.cond.)} \\ [15ex] \hline

\raisebox{-1\height}{
\AxiomC{$|M$}
\noLine
\UnaryInfC{$A$}
\RightLabel{$\vee^+_0$}
\UnaryInfC{$A \vee B$}
\DisplayProof 
 \hspace{10pt}
\AxiomC{$|M$}
\noLine
\UnaryInfC{$B$}
\RightLabel{$\vee^+_1$}
\UnaryInfC{$A \vee B$}
\DisplayProof} & \raisebox{-2.8\height}{$(\vee l_{0,B} M^A)^{A \vee B}$ $(\vee l_{1,A}M^{B})^{A \vee B}$}  \\ [10ex] \hline

\raisebox{-1\height}{
\AxiomC{$|M$}
\noLine
\UnaryInfC{$A \wedge B$}
\AxiomC{$\mathass{u:A}$}
\noLine
\UnaryInfC{$|N$}
\noLine
\UnaryInfC{$C$}
\AxiomC{$\mathass{v:B}$}
\noLine
\UnaryInfC{$|K$}
\noLine
\UnaryInfC{$C$}
\RightLabel{$\vee^- u,v$}
\TrinaryInfC{$C$}
\DisplayProof} & \raisebox{-3.8\height}{$(M^{A \vee B} (u^A.N^{C},v^{B}.K^{C}))^C$} \\ [15ex] \hline
\end{tabular}
\end{center}
\bigskip
\begin{myremark}[Minlog Specifics]
The following is a list of implementation specifics which describe the particular way in which the Minlog system implements program extraction:

\begin{itemize}
\item The disjunction $\vee$ operation is defined as an inductively defined predicate \texttt{ori} with closure axioms instead of  rules.

\item $\exists$ is treated by axioms instead of rules.

\item We also have axioms for the so-called nc-quantifiers (quantifiers without computational content) $\exists^{nc}$ and $\forall^{nc}$. They logically behave like the usual quantifiers ($\exists$ and $\forall$), but they behave differently when it comes 
to program extraction. These nc-quantifiers will be discussed in more detail in the following sections of this chapter.

\item Axioms are represented on the term level by axiom constants.

\end{itemize}

\end{myremark}

\section{Type of a Formula}

In the following section we shall discuss the type of formulas. We shall begin by defining the type of a formula such that if the formula is proven the type corresponds to the type of the extracted program from the proof. Secondly, we shall define the lambda calculus program that can be extracted from a proof term. These proof terms are generated in Minlog by applying the natural deduction calculus to a formula.  Finally we define what it means for a term to realise a formula.

In order to perform program extraction on a proof we must first deduce the
type of the program to be extracted from a formula. To this end, every formula $A$ is assigned a type $\tau(A)$ which corresponds to the type of the program extracted from $A$. It is possible for a formula to have no computation content i.e. $\tau(A) = \epsilon$; such formulas are called \emph{Harrop formulas}.


We will now give a formal definition of $\tau (A)$ starting with the type of a
predicate variable. If the predicate variable has some assignment $\alpha_P$ then the type of that
formula is the value of the assignment. In particular, P  is an inductively defined predicate, then $\alpha_P$ is the algebra whose constructors have the same type as the closure axioms of the inductive predicate. 


\[
\tau(P(\vec{s})) := \left\{ 
\begin{array}{l l}
  \alpha_P & \quad \text{if $P$ is a predicate variable with assigned $\alpha_P$}\\
 \epsilon & \quad \text{otherwise}\\
\end{array} \right.
\]

The type of a formula containing an existential quantifier and a sub-formula
$A$ is dependent on the type of $A$. If $A$ contains no computational content
i.e. $\tau(A) = \epsilon$ then the type of the formula is that of the bound
variable $x$. Otherwise the type is a product of the type of the bound
variable $\rho$ and the type of the formula $A$ $\tau (A)$.

\[
\tau(\exists x^{\rho} A) := \left\{ 
\begin{array}{l l}
  \epsilon & \quad \text{if $\tau(A) = \epsilon$}\\
 \rho \times \tau (A) & \quad \text{otherwise}\\
\end{array} \right.
\]

\[
\tau(\exists^{nc} x^{\rho} A) := 
 \tau (A) 
\]

The type of a formula containing a universal quantifier and a sub-formula $A$
is also dependent on the type of $A$. If $A$ contains no computational content
i.e. $\tau (A) = \epsilon$ then the type of the whole formula is also
$\epsilon$. Otherwise if $A$ does have some computational content then the type
of the formula is a function with $\rho$ as a parameter and returns a result
of type $\tau (A)$.


\[
\tau(\forall x^{\rho} A) := \left\{ 
\begin{array}{l l}
  \epsilon & \quad \text{if $\tau(A) = \epsilon$}\\
 \rho \to \tau (A) & \quad \text{otherwise}\\
\end{array} \right.
\]


The type of a formula containing a non-computational universal quantifier and a sub-formula $A$
is solely dependent on the type of $A$.

\[
\tau(\forallnc x^{\rho} A) := \tau(A)
\]

The type of the formula $A_0 \wedge A_1$ is the product of the types of the two sub-formulas $A_0$ and $A_1$ unless one of the formulae has no computational content. In that case the type of the formula is equal to the sub-formula with computational content.

\[
\tau(A_0 \wedge A_1) := \left\{ 
\begin{array}{l l}
  \tau(A_i) & \quad \text{if $\tau(A_{1-i}) = \epsilon$}\\
 \tau(A_0) \times \tau(A_1) & \quad \text{otherwise}\\
\end{array} \right.
\]

The type of a formula $A \to B$ containing the implication operator and
sub-formulae $A$ and $B$ is dependent on the types of the sub-formulae.
If $\tau (A) = \epsilon$ then the type of the formula is $ \tau (B)$. If $\tau
(B) = \epsilon$ then the type of the formula is $\epsilon$. Otherwise it
returns a function that has a parameter with a type of $\tau (A)$ and returns a value with
a type of $\tau (B)$.

\[
\tau(A \to B) := \left\{ 
\begin{array}{l l}
  \tau (B) & \quad \text{if $\tau (A) = \epsilon$}\\
  \epsilon & \quad \text{if $\tau (B) = \epsilon$}\\
  \tau (A) \to \tau (B) & \quad \text{otherwise}\\
\end{array} \right.
\]

The formula $\bot$ has no computational content.
\[
\tau(\bot) := \epsilon
\]

More details can be found in \cite{HS14}.

\section{Program Extraction}
\label{sec:programextraction}
Given a proof term it is possible to extract a program in the typed lambda calculus  that depends on the type of the term. We shall now give a case by case definition of the program extracted from a formula $\llbracket M \rrbracket$ which has a type $\tau (A)$. Where, $M$ is a derivation of the formula $A$ with $A$ having a defined type i.e. $ \tau (A) \neq \epsilon $.

\[ \llbracket u^A \rrbracket := x^{\tau (A)}_{u} \text{uniquely associated with $u^A$}
\]
\[
\llbracket \lambda u^A M \rrbracket := \left\{ 
\begin{array}{l l}
\llbracket M \rrbracket  & \quad \text{if $\tau (A) = \epsilon$} \\ 
\lambda x^{\tau (A)}_{u} \llbracket M \rrbracket & \quad \text{otherwise} \\
\end{array} \right.
 \]
\[
\llbracket M^{A \to B} N \rrbracket := \left\{ 
\begin{array}{l l}
\llbracket M \rrbracket  & \quad \text{if $\tau (A) = \epsilon$} \\ 
\llbracket M \rrbracket  \llbracket N \rrbracket & \quad \text{otherwise} \\
\end{array} \right.
 \]
\[
\llbracket \langle M^{A_0}_0, M^{A_1}_1 \rangle \rrbracket := \left\{ 
\begin{array}{l l}
\llbracket M_i \rrbracket  & \quad \text{if $\tau (A_{1-i}) = \epsilon$} \\ 
\langle \llbracket M_0 \rrbracket , \llbracket M_1 \rrbracket \rangle & \quad \text{otherwise} \\
\end{array} \right.
 \]
$$\llbracket ( \lambda x^{\rho} M)^{\forall x A} \rrbracket := \lambda x^{\rho} \llbracket   M \rrbracket$$
$$\llbracket M^{\forall x A} t \rrbracket := \llbracket M \rrbracket t$$
$$ \llbracket (\lambda x^{\rho} M)^{\forallnc x A} \rrbracket := \llbracket M \rrbracket $$
$$ \llbracket M^{\forallnc x A} t \rrbracket := \llbracket M \rrbracket  $$

For derivations $M^A$ with type $\tau(A) = \epsilon$, we define the extracted program to be $\llbracket M \rrbracket := \epsilon$, where $\epsilon$ is some new symbol. 

The axioms for dealing with existential quantifiers produce slightly more
complicated programs upon extraction. 

$$ \exists^+_{x,A} : \forall x^{\rho}.A \to \exists x^{\rho} A  \ \mathbf{ex \mhyphen intro}$$
$$ \exists^-_{x,A,B} : \exists x^{\rho} A \to (\forall x^{\rho}. A \to B) \to
B  \ \mathbf{ex \mhyphen elim}$$

The following are the extracted programs for the above axioms.

\[ \llbracket \exists^{+}_{x^{\rho}, A} \rrbracket := \left\{ 
\begin{array}{l l}
\lambda x^{\rho} x  & \quad \text{if $\tau (A) = \epsilon$} \\ 
\lambda x^{\rho} \lambda y^{\tau (A)} \langle x,y \rangle & \quad \text{otherwise} \\

\end{array} \right.
\]

There are different induction schemes  that yield different extracted programs however in general they yield a recursion operation. The extracted program $\llbracket ind_j \rrbracket$ is defined to be the recursion operation $\mathcal{R}^{\vec{\mu},\vec{\tau}}_{\mu_j}$


\[ \llbracket \exists^{-}_{x^{\rho}, A, B} \rrbracket := \left\{ 
\begin{array}{l l}
\lambda x^{\rho} \lambda f^{\rho \to \tau (B)}. f x  & \quad \text{if $\tau (A) = \epsilon$} \\ 
\lambda z^{\rho \times \tau (A)} \lambda f^{\rho \to \tau (A) \to \tau (B)}. f
(z0) (z1) & \quad \text{otherwise} \\
\end{array} \right.
\]


Induction on an inductively defined predicate corresponds on the program level  to recursion on the corresponding algebra. The closure axioms of an inductively defined predicate are realized by the constructors of the algebra. The extracted program for an application of general induction is general recursion which in Minlog is called guarded recursion and implemented by the operation \texttt{GRecGuard}. During our later proof we perform induction on a measure and as a result this same measure also appears in the extracted Minlog term.  This guarded recursion term can be expressed as a function as follows:

\begin{lstlisting}[caption = The Guarded Recursion Operation Used in the Minlog System, mathescape]
(GRecGuard $\rho$1 $\rho$2 $\rho$3 $\tau$) $\mu$ z1 z2 z3 G t = f z1 z2 z3 t where
f : $\rho$1 $\implies$ $\rho$2 $\implies$ $\rho$3 $\implies$ Boole $\implies$ $\tau$
f x1 x2 x3 True = G x x2 x3 (
        [y1, y2, y3].f y1 y2 y3($\mu$ y1 y2 y3 < $\mu$ x1 x2 x3))
f x1 x2 x3 False = Inhab
\end{lstlisting}


Similarly, the extracted program for non-measure induction is a recursion operation \texttt{Rec}. It is possible to write the operation \texttt{Rec list} $\rho$ $\implies$ $\tau$ using a function
f : list $\rho \implies \tau$ which gives a more intuitive view of its behaviour.
\begin{lstlisting}[caption = The Recursion Operation Used in the Minlog System ,mathescape]
(Rec list $\rho \implies \tau$)ys z G = f ys where
f Nil = z
f (x :: xs) = G x xs(f xs)
\end{lstlisting}


When program extraction is applied to an inductive definition a data structure is extracted. Inductive definitions are assigned algebras 
where the types of constructors of the algebras are determined by the types of the closure axioms.
\section{Realisability}\label{sec:realisability}
We need to have a notion of what it means for an extracted program to realise
a formula. The notion that we define to do this is called realisability which takes shape as a formula $r \textbf{r} A$. This can be read as
a term $r$ realises a formula $A$. If $\tau(A) \neq \epsilon$ then $r$ is of type $\tau (A)$,
otherwise $r$ is the symbol $\epsilon$. We also need to define realisability
for predicates. If we have a predicate $P$
with an arity $\vec{s}$ then there are two possibilities here. One is that
$P$ is a predicate variable with an assigned $\alpha_P$ for this we create a new
predicate variable $P^{\textbf{r}}$ with an arity of $\alpha_P,\vec{\rho}$. In Minlog this corresponds to inductive definitions being realised algebras.
The other case is that $P$ is a predicate constant, if this is the case then
we use the predicate constant.



\[ r \ \textbf{r} \ P(\vec{s}) = \left\{ 
\begin{array}{l l}
P^r(r,\vec{s}) & \quad \text{if $P$ is a predicate variable with assigned $\alpha_P$} \\ 
P(\vec{s}) & \quad \text{if P is a predicate constant} \\

\end{array} \right.
\]

\[ r \ \textbf{r} \ (\exists x A) = \left\{ 
\begin{array}{l l}
\epsilon \ \textbf{r} \ A[x:=r] & \quad \text{if $\tau (A) = \epsilon$} \\ 
r1 \ \textbf{r} \ A[x:= r0] & \quad \text{otherwise} \\

\end{array} \right.
\]

\[ r \ \textbf{r} \ (\forall x A) = \left\{ 
\begin{array}{l l}
\forall x. \epsilon \ \textbf{r} \ A & \quad \text{if $\tau (A) = \epsilon$} \\ 
\forall x. rx \ \textbf{r} \ A & \quad \text{otherwise} \\

\end{array} \right.
\]


\[ r \ \textbf{r} \ (\exists^{nc} x A) = \left\{ 
\begin{array}{l l}
\exists^{nc} x. \epsilon \ \textbf{r} \ A & \quad \text{if $\tau (A) = \epsilon$} \\ 
\exists^{nc} x. r \ \textbf{r} \ A & \quad \text{otherwise} \\

\end{array} \right.
\]

\[ r \ \textbf{r} \ (\forall^{nc} x A) = \left\{ 
\begin{array}{l l}
\forall x. \epsilon \ \textbf{r} \ A & \quad \text{if $\tau (A) = \epsilon$} \\ 
\forall x. r \ \textbf{r} \ A & \quad \text{otherwise} \\

\end{array} \right.
\]
\[ r \ \textbf{r} \ (A \to B) = \left\{ 
\begin{array}{l l}
\epsilon \ \textbf{r} \ A \to r \ \textbf{r} \ B & \quad \text{if $\tau (A) = \epsilon$} \\ 
\forall x.x \ \textbf{r} \ A \to \epsilon \ \textbf{r} \ B & \quad \text{if
  $\tau (A) \neq \epsilon \neq \tau (B)$} \\
\forall x. x \ \textbf{r} \ A \to r x  \ \textbf{r} \ B & \quad \text{otherwise} \\

\end{array} \right.
\]

\[ r \ \textbf{r} \ (A \wedge B) = \left\{ 
\begin{array}{l l}
\epsilon \ \textbf{r} \ A  \wedge r \ \textbf{r} \ B & \quad \text{if $\tau (A) = \epsilon$} \\ 
r \ \textbf{r} \ A \wedge \epsilon \ \textbf{r} \ B & \quad \text{if
  $\tau (B) \neq \epsilon$} \\
r_1 \ \textbf{r} \ A \wedge r_2  \ \textbf{r} \ B & \quad \text{otherwise} \\

\end{array} \right.
\]



\medskip

\begin{mytheorem}
\emph{(Soundness Theorem)}

If $M$ is a $\mathrm{HA}^{\mu}$ proof of a closed formula $B$, then we can derive  $ \llbracket M \rrbracket \ \textbf{r} \ B$ in $\mathrm{HA}^{\mu}$.

\begin{proof}

The proof proceeds by performing induction on the proof term $M$, we omit further details (See \cite{HS14} 4.3.6).

\begin{comment}
\begin{description}

\item[Case] $\lambda u^A M^B$ We must derive $$\llbracket \lambda u^A M^B
  \rrbracket \textbf{r} B$$
   \begin{description}
     \item[Subcase] $\tau(A) = \epsilon$ In this case the empty term realises
       A, making the $\lambda u $ redundant, which implies that the extracted
       program from M realises B $$\llbracket \lambda u M \rrbracket
       \textbf{r} (A \to B) = \epsilon \textbf{r} A \to \llbracket M
       \rrbracket \textbf{r} B$$
     \item[Subcase] $\tau(A) \neq \epsilon = \tau (B)$ In this case the
       extracted program is empty $ \llbracket \lambda u M \rrbracket = \epsilon$.

       $$\llbracket \lambda u M \rrbracket \textbf{r} (A \to B) = \forall
       x. x \textbf{r} A \to \epsilon \textbf{r} B$$
     \item[Subcase]$\tau (A) \neq \epsilon \neq \tau (B)$ neither the type of $A$ or $B$ is empty. Therefore if we have an $x$ that
       realises $A$ then this $x$ applied to the extracted program realises B.
         

   \end{description}

\item[Case] $\exists^+_{x,A}$ We now consider the case that an existential
  introduction axiom had been applied in the proof. The axiom takes the
  following form $$\exists^+_{x,A}: \forall^{nc} \vec{\rho} \forall x. A \to
  \exists x A $$
  For our proof of the soundness theorem we need to find a derivation
  of $$\llbracket \exists^+_{x,A} \rrbracket \textbf{r} \forall^{nc}
  \vec{\rho} \forall x.A \to \exists x A$$

  \begin{description}
   
   \item[Subcase] $\tau (A) = epsilon$ If the type of A is empty then the
     extracted program is $\lambda x x$ which realises the formula
     corresponding to the existential introduction axiom.

     $$(\lambda x x) \textbf{r}  \forall^{nc} \vec{\rho} \forall x. A \to
  \exists x A  =  \forall^{nc} \vec{\rho} \forall x.x \textbf{r} (A \to
  \exists x A) $$

  Since $\tau (A) = \epsilon$ the empty program realises A.
  $$\forall^{nc} \vec{\rho} \forall x.x \textbf{r} (A \to
  \exists x A )  = \forall^{nc} \vec{\rho} \forall x.\epsilon \textbf{r} A \to
  x \textbf{r} \exists x A $$

  $$\forall^{nc} \vec{\rho} \forall x.\epsilon \textbf{r} A \to
  x \textbf{r} \exists x A = \forall^{nc} \vec{\rho} \forall x.\epsilon \textbf{r} A \to
  \epsilon \textbf{r}  A $$

   \item[Subcase] $\tau (A) \neq \epsilon$ If the type of A is non empty then
     the extracted program is $\lambda x \lambda f \langle x ,f \rangle$ which
     realises the formula corresponding to the existential introduction axiom.

     $$ \lambda x \lambda f \langle x ,f \rangle \textbf{r}  \forall^{nc} \vec{\rho} \forall x. A \to
  \exists x A    \forall^{nc} \vec{\rho} \forall x,f. f \textbf{r} A \to
   \langle x ,f \rangle \textbf{r} \exists x A $$

       $$ \forall^{nc} \vec{\rho} \forall x,f. f \textbf{r} A \to
   \langle x ,f \rangle \textbf{r} \exists x A  = \forall^{nc} \vec{\rho} \forall x,f. f \textbf{r} A \to
   f \textbf{r} A  $$
  \end{description}

  \item[Case] $\exists^-_{x,A,B}$  We now consider the case that an
    existential elimination axiom had been applied in the proof. The axiom
    takes the following form $$\exists^-_{x,A,B}: \forall^{nc}
    \vec{\rho}.\exists x A \to (\forall x. A \to B) \to B $$ for the proof of
    our theorem we need to find a derivation of $$\llbracket
    \exists^-_{x,A,B} \rrbracket \textbf{r} \forall^{nc} \vec{\rho}. \exists
    x A \to (\forall x. A \to B) \to B$$
    \item[Subcase] $\tau (A) = \epsilon = \tau (B)$ The extracted program
      in this case is the empty program that is $ \llbracket \exists^-_{x,A,B}
      \rrbracket = \epsilon $

      $$\epsilon \textbf{r} \forall^{nc} \vec{\rho}. \exists
       x A \to (\forall x. A \to B) \to B 
      = \forall^{nc} \vec{\rho} \forall x. x \textbf{r} \exists x A \to \epsilon \textbf{r}((\forall x. A \to B) \to B) 
      = \forall^{nc} \vec{\rho} \forall x. \epsilon \textbf{r} A \to
      \epsilon \textbf{r}(\forall x. A \to B) \to \epsilon \textbf{r} B  
      = \forall^{nc} \vec{\rho} \forall x. \epsilon \textbf{r} A \to
      \forall x \epsilon \textbf{r}(A \to B) \to \epsilon \textbf{r} B
      = \forall^{nc} \vec{\rho} \forall x. \epsilon \textbf{r} A \to
      (\forall x. \epsilon \textbf{r} A \to \epsilon \textbf{r} B) \to
      \epsilon \textbf{r} B$$

    \item[Subcase] $\tau (A) \neq \epsilon = \tau(B)$ Once again the
      extracted program in this case is the empty program that is $ \llbracket \exists^-_{x,A,B}
      \rrbracket = \epsilon $




    \item[Subcase] $\tau (A) = \epsilon \neq \tau(B)$
    \item[Subcase] $\tau (A) \neq \epsilon \neq \tau(B)$


\end{description}
\end{comment}

\end{proof}

\end{mytheorem}


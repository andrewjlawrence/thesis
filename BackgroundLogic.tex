\mathchardef\mhyphen="2D
\newcommand{\mathass}[1]{\mathrm{[}#1 {]}}
\chapter{Background: Introduction to Logic}
We use logic to formally reason about the world around us in an abstract way. At the most basic level we are creating abstract propositions, sentences from these propositions and then using formal rules to deal with these propositions and to derive new information or insight. In the following we give a general introduction to logic and describe some of the formalisms used in this thesis.
A good general introduction to logic from a proof theoretic perspective is \cite{HandBookProof}.


\section{Propositional logic}
One of the most fundamental forms of logic is propositional logic which contains operations called connectives that allows one to connect propositional variables to form sentences.


\begin{mydef}[Propositional Formulae]
Given a set of propositional variables $V$ we inductively define the set of propositional formulae $AP$ as follows:
\begin{itemize}
\item If $s$ is in the set of symbols $s \in V$ then $s \in AP$.

\item if $p_1$ is a propositional formula $p_1 \in AP$ then $\neg p_1 \in AP$.

\item given two propositional formulae $p_1$ and $p_2$ then $p_1 \circ p_2 \in AP$ where $\circ$ is a propositional connective $\circ \in \{ \wedge,\vee,\to \} $.
\end{itemize} 
\end{mydef}
In order to define the semantics of propositional logic we need the notiona of a truth assignment which is a function $\tau : \{True,False\}^n \to \{True, False\}$ that gives a logical assignment to the set propositional variables $V$ which is of size $n$. This function can then be used to describe another function $\bar{\tau} : P \to \{True, False\}$ that given a propositional formula $P \in AP$ calculates a truth assignment for that formula based on $\tau$.

We assigned these sentences of propositions a value of either True or False using a truth assignment 
\begin{center}
\begin{tabular}{c | c | c | c | c | c | c} \hline
$A$ & $B$ & $\neg A$  & $A \wedge B$ & $A \vee B$ & $A \to B$ & $A \leftrightarrow B$  \\ \hline
 T & F & F  & T & T & T & T \\ \hline 
 T & T & F  & F & T & F & F \\ \hline
 F & F & T  & F & T & T & F \\ \hline
 F & T & T  & F & F & T & T \\ 
\end{tabular}
\end{center}

\section{First Order Logic}
\begin{mydef}[Syntax of First Order Logic]
Blah

\end{mydef}

\begin{mydef}[Semantics of First Order Logic]

\end{mydef}

\section{Linear Temporal Logic}
In order to perform model checking over a system we typically need a formal language that allows one to speak about time. We need to be able to formalise sentences such as "the next moment of time" and "all moments in time in the future". One such logic that allows us to formalise these statements is linear temporal logic (LTL)\cite{AP77}. 


Using these atomic propositions combined with the temporal operators it is possible to define a syntax for temporal logic formulae.

\begin{mydef}[Syntax of Linear Temporal Logic]
Let $AP$ be the set of propositional formulae then:

\begin{itemize}
\item $\top$ and $\bot$ are well formed formulas.
\item if $p \in AP$ then $p$ is well formed formula (wff).

\item if $f$ and $g$ are wff  then $\star f$ and $f \circ g$ are wffs where $\star \in \{\neg,\mathbf{X},\mathbf{G}, \mathbf{F}\}$ and $\circ \in \{ \wedge,\vee,\textbf{R},\textbf{U} \}$.
\end{itemize}

\end{mydef}
We write $LTL(AP)$ to denote the LTL logic formulas for a given set of atomic propositions $AP$.

LTL operations can be used to speak about paths through a system specified as a Kripke structure.
a \emph{path} $\pi$ is a sequence of states $s_1, \ldots s_n$ and a \emph{path formula} is one that holds in each given state of a path. We define $Path(K,s_0)$ to be the set of paths starting at state $s_0$ in the Kripke structure $K$ as the set of functions $\phi : N \to S$ such that $\phi(0) = a$ and the $\phi(n) \to \phi(n+1)$. 

We shall now look at the semantics of LTL firstly using an informal description of the LTL operations and secondly by giving a formal semantics for LTL. The following is a description of the 5 LTL operations over paths of a Kripke Structure.

\begin{itemize}
\item \textbf{X} $f$ : The property $f$ holds in the \emph{next} moment of time.
\item \textbf{G} $f$ : The property $f$ is \emph{globally} true. i.e. it holds for all times on all paths. 
\item \textbf{F} $f$ : The property $f$ is \emph{finally} true. i.e. there exists a time such that the property $f$ holds on a path.
\item $f$ \textbf{U} $g$ : For all paths the property globally $f$ holds \emph{until} property $g$ holds. 
\item $f$ \textbf{R} $g$ : f holds up to and including the point when $g$ holds.
\end{itemize}

%%%
%%% We need to formalise the following definition. Semantics of Linear Temporal Logic Formula?
%%% 
The semantics of LTL is defined inductively in terms of the $\mathbf{X}$ and $\mathbf{U}$ LTL operations which can then be used in combination with operator equivalences to define the semantics of other operations.
 
\begin{mydef}[Semantics of Linear Temporal Logic]
We define the semantics of a linear temporal logic formula $\phi \in LTL(AP)$ in terms of a satisfaction relation $$K,s \models \phi$$ over a Kripke structure $K = (S,T,L)$ and a state $s \in S$ as follows:

$$K,s \models \phi$$ holds if and only if $\forall \pi \in Path(K, s)$, $K,\pi \models \phi$ holds.

We define what it means for $K, \pi \models \phi$ to hold inductively as follows:

\begin{itemize} 
\item $K,\pi \models \phi$ always holds.

\item For all propositional formulae $p \in AP$ the following always holds:
     $$K,\pi \models p \leftrightarrow p \in L(\pi(0))$$

\item For all LTL formulas $\mathbf{X} \phi \in LTL(K)$ the following always holds:
$$ K, \pi \models \mathbf{X} \phi \leftrightarrow K, succ;\pi \models \phi $$

where $succ$ is a successor function $succ: N \_ \to N$ such that $succ;\pi(n) = \pi(succ(n)) = \pi(n + 1)$.

\item For all LTL formulas $\phi \mathbf{U} \psi \in LTL(K)$ the following always holds:
$$K, \pi \models_{LTL} \phi \mathbf{U} \psi \leftrightarrow$$
$$\exists n \in N.$$
$$(K,succ^n; \pi \models_{LTL} \psi) \wedge (\forall m \in N. m < n \to K, succ^m;\i \models_{LTL} \phi)$$

\item For all LTL formulas $\neg \phi \in LTL(AP)$ the following always holds:
$$K,\pi \models_{LTL} \neg \phi \leftrightarrow K,\pi \not\models \phi $$

\item For all LTL formulas $\phi \vee \psi \in LTL(AP)$ the following always holds
$$K,\pi \models_{LTL} \phi \vee \psi \leftrightarrow K,\pi \models_{LTL} \phi \ \mathrm{or} \ K,\pi \models_{LTL} \psi $$

\end{itemize}

\end{mydef}
In the above definition we have defined the semantics for the \textbf{U} and \textbf{X} LTL operations. 
The semantics for formulas containing other LTL operations can be obtained by using the following equivalences:
$$\textbf{G} \phi \equiv False \mathbf{R} \phi $$
$$\textbf{F} \phi \equiv  True \mathbf{U} \phi$$

Combined  with the equivalences for obtaining negative normal form.

\begin{mydef}[LTL Negative Normal Form]
The negative normal form of an LTL formula can be obtained by applying the following equivalences:
$$\neg True \equiv False$$
$$\neg False \equiv True$$
$$\neg \neg \phi \equiv \phi$$
$$\neg (\phi \vee \psi) \equiv \neg \phi \wedge \neg \psi $$
$$\neg (\phi \wedge \psi) \equiv \neg \phi \vee \neg \psi$$
$$\neg \mathbf{X} \phi \equiv \mathbf{X} \neg \phi$$
$$\neg (\phi \mathbf{U} \psi) \equiv (\neg \phi) \mathbf{R} (\neg \psi)$$
$$\neg (\phi \mathbf{R} \psi \equiv (\neg \phi) \mathbf{U} (\neg \psi)$$
from left to right until no further equivalence is applicable.
\end{mydef}

\section{Proof Systems}

\section{Natural Deduction}
The natural deduction proof system was introduced by Gentzen in 1939 \cite{GG69}. He was looking to capture the method by which humans naturally reason in a logical formalism.  The resulting system is very general and allows for new facts to be deduced from what is previously known in an intuitive manor. 
\begin{mydef}[Natural Deduction] 
The natural deduction proof system consists of the following rules:
\begin{center}
\AxiomC{$A$}
\AxiomC{$B $}
\LeftLabel{$\wedge \mhyphen intro$}
\BinaryInfC{$A \wedge B$}
\DisplayProof \hspace{10pt}
\AxiomC{$A \wedge B$}
\LeftLabel{$\wedge \mhyphen elim \, (L)$}
\UnaryInfC{$A$}
\DisplayProof \hspace{10pt}
\AxiomC{$A \wedge B$}
\LeftLabel{$\wedge \mhyphen elim \, (R)$}
\UnaryInfC{$A$}
\DisplayProof
\end{center}
\medskip

\begin{center}
\AxiomC{$A$}
\LeftLabel{$\vee \mhyphen intro \, (L)$}
\UnaryInfC{$A \vee B$}
\DisplayProof \hspace{10pt}
\AxiomC{$B$}
\LeftLabel{$\vee \mhyphen intro \, (R)$}
\UnaryInfC{$A \vee B$}
\DisplayProof \\
\end{center} 
\bigskip
\begin{center}
\AxiomC{$A \wedge B$}
\AxiomC{$\mathass{A}$}
\noLine
\UnaryInfC{\vdots}
\noLine
\UnaryInfC{$C$}
\AxiomC{$\mathass{B}$}
\noLine
\UnaryInfC{\vdots}
\noLine
\UnaryInfC{$C$}
\LeftLabel{$\vee \mhyphen elim$}
\TrinaryInfC{$C$}
\DisplayProof
\end{center}
\medskip
\begin{center}
\AxiomC{$\mathass{A}$}
\noLine
\UnaryInfC{$\vdots$}
\noLine
\UnaryInfC{$B$}
\LeftLabel{$\to \mhyphen intro $}
\UnaryInfC{$A \vee B$}
\DisplayProof \hspace{10pt}
\AxiomC{$A \to B$}
\AxiomC{$A$}
\LeftLabel{$\to \mhyphen elim$}
\BinaryInfC{$B$}
\DisplayProof
\end{center}
\medskip
\begin{center}
\AxiomC{$\bot$}
\LeftLabel{efq}
\UnaryInfC{$A$}
\DisplayProof 
\end{center}
\medskip
\begin{center}
\AxiomC{$(a/x)A$}
\LeftLabel{$\forall \mhyphen intro$}
\UnaryInfC{$\forall x. A$}
\DisplayProof \hspace{10pt}
\AxiomC{$\forall x. A$}
\LeftLabel{$\forall \mhyphen elim$}
\UnaryInfC{$(t/x) A$}
\DisplayProof
\end{center}
\medskip
\begin{center}
\AxiomC{$\exists x. A$}
\AxiomC{$\mathass{(a/x)A}$}
\noLine
\UnaryInfC{$\vdots$}
\noLine
\UnaryInfC{$C$}
\LeftLabel{$\exists \mhyphen elim$}
\BinaryInfC{$C$}
\DisplayProof \hspace{10pt}
\AxiomC{$(a/x) A$}
\LeftLabel{$\exists \mhyphen intro$}
\UnaryInfC{$\exists x. A$}
\DisplayProof
\end{center}

\end{mydef}